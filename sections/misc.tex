\section{Miscellaneous}
\label{sec:misc}
Features/Phenomenon which didn't fit into any other category.


\subsection{Influence of movement keys}
\label{sec:movementkeys}
In most instances where $\fmove$ and $\rmove$ are involved, current velocity is first projected down onto the $xy$-plane and normalized.
\codeFromFile{firstline=621,lastline=625,gobble=1}{code/game/bg_pmove.c}

Using \eqref{eq:cmd} and \eqref{eq:fm_rm_um},

\begin{align*}
\tuvec*{\flat{\fmove}} &=
\begin{pmatrix}
	\cos\rho\cos\gamma\\
	\cos\rho\sin\gamma\\
	0
\end{pmatrix} \frac{1}{\sqrt{\cos^2\rho}} =
\begin{pmatrix}
	\cos\gamma\\
	\sin\gamma\\
	0
\end{pmatrix},\\
\tuvec*{\flat{\rmove}} &=
\begin{pmatrix}
	-\sin\sigma\sin\rho\cos\gamma + \cos\sigma\sin\gamma\\
	-\sin\sigma\sin\rho\sin\gamma - \cos\sigma\cos\gamma\\
	0
\end{pmatrix} \frac{1}{\sqrt{\sin^2\sigma\sin^2\rho + \cos^2\sigma}}.
\end{align*}

Finding the dot product,
\begin{align*}
\tuvec*{\flat{\fmove}} \cdot \tuvec*{\flat{\rmove}} &= \tuvec*{\flat{\fmove}}^T \tuvec*{\flat{\rmove}} = \frac{-\sin\sigma\sin\rho\cos^2\gamma + \cos\sigma\sin\gamma\cos\gamma - \sin\sigma\sin\rho\sin^2\gamma - \cos\sigma\sin\gamma\cos\gamma}{\sqrt{\sin^2\sigma\sin^2\rho + \cos^2\sigma}}\\
&= \frac{-\sin\sigma\sin\rho}{\sqrt{\sin^2\sigma\sin^2\rho + \cos^2\sigma}}.
\end{align*}

When in air or on flat ground, wishvel is equated as
\begin{align*}
\texttt{wishvel} &= \fmove \tuvec*{\flat{\fmove}} + \rmove \tuvec*{\flat{\rmove}}\\
&= \tvec{\flat{\fmove}} + \tvec{\flat{\rmove}}
\end{align*}

\begin{align*}
\texttt{wishvel} &= \fmove
\begin{pmatrix}
	\cos\gamma\\
	\sin\gamma\\
	0
\end{pmatrix} + \rmove
\begin{pmatrix}
	-\sin\sigma\sin\rho\cos\gamma + \cos\sigma\sin\gamma\\
	-\sin\sigma\sin\rho\sin\gamma - \cos\sigma\cos\gamma\\
	0
\end{pmatrix} \frac{1}{\sqrt{\sin^2\sigma\sin^2\rho + \cos^2\sigma}},
\end{align*}

\begin{align*}
\norm{\texttt{wishvel}}^2 = &\fmove^2 \cos^2\gamma + \rmove^2 \frac{\sin^2\sigma\sin^2\rho\cos^2\gamma + \cos^2\sigma\sin^2\gamma - \sin\sigma\cos\sigma\sin\rho\sin\gamma\cos\gamma}{\sin^2\sigma\sin^2\rho + \cos^2\sigma} +\\
&\fmove^2 \sin^2\gamma + \rmove^2 \frac{\sin^2\sigma\sin^2\rho\sin^2\gamma + \cos^2\sigma\cos^2\gamma + \sin\sigma\cos\sigma\sin\rho\sin\gamma\cos\gamma}{\sin^2\sigma\sin^2\rho + \cos^2\sigma} +\\
&2\fmove\rmove\frac{-\sin\sigma\sin\rho\cos^2\gamma + \cos\sigma\sin\gamma\cos\gamma}{\sqrt{\sin^2\sigma\sin^2\rho + \cos^2\sigma}} +\\
&2\fmove\rmove\frac{-\sin\sigma\sin\rho\sin^2\gamma - \cos\sigma\sin\gamma\cos\gamma}{\sqrt{\sin^2\sigma\sin^2\rho + \cos^2\sigma}},\\
%
= &\fmove^2 + \rmove^2 + 2\fmove\rmove\frac{-\sin\sigma\sin\rho}{\sqrt{\sin^2\sigma\sin^2\rho + \cos^2\sigma}},
\end{align*}

which has the derivative
\begin{align*}
\frac{\diff}{\diff\rho} \norm{\texttt{wishvel}}^2 = 2\fmove\rmove\frac{-\sin\sigma\cos^2\sigma\cos\rho}{\left( \sin^2\sigma\sin^2\rho + \cos^2\sigma \right)^{3/2}}.
\end{align*}
This is $0$ when
\[
\begin{cases}
	\rho = \hphantom{-}\frac{\pi}{2} + 2k\pi, &k \in \symbb{N}\\
	\rho = -\frac{\pi}{2} + 2k\pi, &k \in \symbb{N}.
\end{cases}
\]

With optimal $\rho$ filled in, we get
\begin{align*}
\tuvec*{\flat{\fmove}}^T \tuvec*{\flat{\rmove}} &= -\sin\sigma,\\
&= \sin\sigma,
\end{align*}
\begin{align*}
\norm{\texttt{wishvel}}^2 &= \fmove^2 + \rmove^2 - 2\fmove\rmove\sin\sigma,\\
&= \fmove^2 + \rmove^2 + 2\fmove\rmove\sin\sigma.
\end{align*}

Appendix \ref{app:cmd_scale} gives rise to the formulae
\begin{align}
\label{eq:scale}
\texttt{scale} = \frac{\texttt{g\_speed}\norm{\tvec{\cmd}}_{\infty}}{127\norm{\tvec{\cmd}}},
\end{align}
\begin{align*}
\texttt{wishspeed} &= \texttt{scale} \norm{\texttt{wishvel}} =
\frac{\texttt{g\_speed}\norm{\tvec{\cmd}}_{\infty}}{127\norm{\tvec{\cmd}}} \norm{\texttt{wishvel}},\\
\norm*{\vec{\flat{a}}} &= \flat{a} = AT \texttt{wishspeed} = AT\frac{\texttt{g\_speed}\norm{\tvec{\cmd}}_{\infty}}{127\norm{\tvec{\cmd}}} \norm{\texttt{wishvel}}.
\end{align*}
When $\fmove = 127$, $\rmove = 0$ and $\umove = 0$, then $s = 320$ like in the previous sections.\\
This is also the case when $\fmove = 127$, $\rmove = 127$ and $\umove = 0$.

Air:
\begin{align}
\nonumber
\norm*{\projmat{a} \vec{\flat{v}}_f} + \norm*{\vec{\flat{a}}} &= s \norm*{\uvec{\flat{a}}},\\
\nonumber
\vec{v}_{f}^T\texttt{wishdir} + \texttt{accel}T\texttt{wishspeed} &= \texttt{wishspeed}\norm{\texttt{wishdir}},\\
\nonumber
\norm*{\vec{\flat{v}}_f} \cos{\delta_{\opt}} + \texttt{accel}T\texttt{wishspeed} &= \texttt{wishspeed},\\
\nonumber
\flat{v}_f \cos{\delta_{\opt}} &= \texttt{wishspeed} - \texttt{accel}T\texttt{wishspeed},\\
\delta_{\opt} &= \acos\left( \frac{\texttt{wishspeed} \left(1 - \texttt{accel}T\right)}{\flat{v}_f} \right).
\end{align}

\texttt{PM\_Accelerate (wishdir, wishspeed, pm\_airaccelerate);}

The strafe keys in CPM do not ever affect upmove. Not true


\subsection{Upmove effect on acceleration}
As mentioned in Section \ref{sec:movementkeys}, upmove has an effect on acceleration while in air.
\label{sec:upmove}
\codeFromFile{firstline=615,lastline=634,gobble=1}{code/game/bg_pmove.c}
From the above code listing, before acceleration is applied the \texttt{wishspeed} ($s$) is multiplied by a scale factor.
This scale factor is determined by Appendix \ref{app:cmd_scale}. In combination with the above code, the \texttt{PM\_CmdScale} function results in the player inputs being normalized then projected down onto the $xy$-plane, then the magnitude is multiplied by $320$ (\texttt{g\_speed}).
From equation \eqref{eq:scale}, in air this produces
\begin{align}
\nonumber
s &= 320\frac{\max(\fmove, \rmove, \umove)}{127\sqrt{\norm*{\tvec{\fmove}}^2 + \norm*{\tvec{\rmove}}^2 + \norm*{\tvec{\umove}}^2}}\sqrt{\norm*{\fmove \tuvec*{\flat{\fmove}}}^2 + \norm*{\rmove \tuvec*{\flat{\rmove}}}^2}\\
\label{eq:s_in_air}
&= \frac{320\norm*{\tvec{\flat{\cmd}}}}{\norm*{\tvec{\cmd}}}.
\end{align}
Hence upmove increases the denominator of the scale factor, decreasing \texttt{wishspeed}.\\
If the player is holding 3 movement keys ($\fmove$, $\rmove$ and $\umove$), $s = 320\frac{\sqrt{2}}{\sqrt{3}} = \qty{261.279}{ups}$. This case is demonstrated in Figure \ref{fig:fm_rm_um_angled_view}.\\
If the player is holding 2 movement keys ($\fmove$ and $\umove$ or $\rmove$ and $\umove$), $s = 320\frac{\sqrt{2}}{2} = \qty{226.274}{ups}$.

Hence, the snap zones will change along with all CGazHUD $\delta$ angles, which explains the altered acceleration while holding moveup or movedown.
Also, holding 2 movement keys reduces $\texttt{wishspeed}$ moreso than holding 3 movement keys, decreasing acceleration more\footnote{Thus, although insignificant, full-beat VQ3 strafing is superior to half-beat and moreso to invert.}.


\subsection{Frame rate dependency}
\label{sec:framerate}
We can define \emph{frame rate dependency} as attributes of the game engine that act differently depending on frame rate,
and say that an attribute of the game engine has \emph{frame independency} if the time between frames does not need to be equidistant for the physics to act the same.\\

Given that acceleration is impacted directly by frame rate (by $\flat{a} = sAT$), frame rate has a large impact on speed gains from strafing.

TODO: frame rate effect on strafing\\

Additionally, $T$ is floored to the nearest millisecond (0.001), giving us proper-T $T_p$, and frames are delayed until $1000T_p$ milliseconds pass.
For example, with frame rate $\qty{140}{fps}$, $T$ would be floored from $1/140 = 0.007143$ to 0.007. Note: \texttt{minMsec} is an int.
\codeFromFile{firstline=2686,lastline=2690,gobble=1}{code/qcommon/common.c}
\codeFromFile{firstline=2599,lastline=2625,gobble=1}{code/qcommon/common.c}
Hence,
\begin{align*}
T_p &=
\begin{cases}
\min\left(5, \max\left(0.001, \dfrac{\floor{1000T}}{1000} \right) \right), & \text{when running a dedicated server or client of a remote server}\\
\min\left(0.2, \max\left(0.001, \dfrac{\floor{1000T}}{1000} \right) \right), & \text{otherwise.}
\end{cases}
\end{align*}
Also, frame rate impacts vertical ($z$-axis) velocity when in air, shown by
\begin{align*}
r_z &= v_z - gT_p\ &&\ldots\ g = 800,\\
\round{r_z} &= v_z - \round*{gT_p}\ &&\ldots\ v_z = \round{v_z}\ \text{when in air}
\end{align*}
Hence, with $T = 1/125 = T_p$,
\begin{align*}
r_z &= v_z - 6.4,\\
\round{r_z} &= v_z - 6,
\end{align*}
and thus over one second $v_z$ is reduced by $\qty{750}{ups}$ instead of $\qty{800}{ups}$, allowing the player to jump higher and hence further.
Although not entirely correct, as a general rule if effective-gravity $\frac{\round*{gT_p}}{T_p} < 800$ then frame rates given by $\frac{1}{T}$ allow the player to jump higher than intended.
This is expanded upon in Section \ref{sec:x_z}.

All times $T$ has been mentioned in this paper should thus be replaced with $T_p$.\\

Another important point to note is that accelerations $a_n$ on an axis $n$ can be rounded to $0$ if $a_n\le 0.5$.
This ironically places limits on the functionality of the movement physics at high frame rates, as accelerations will be more frequently nulled.
% TODO: Also need to find out what happens in a situation where the flooring to T_p causes the game to try to render more frames per second than FPS(does it?)
% Eg. 120 FPS. int minMsec = 1000 / 120 = 8.3333, floors to 8. However, T=8 implies 125 frames rendered per second, yet we only have 120 FPS.

Frame rate dependency with respect to overbounces is covered in Section \ref{sec:overbounce} (TODO).


\subsection{Approach to $z$-axis translations}
\label{sec:x_z}
The game engine applies vertical velocity in a slightly unintuitive way in order to make all $z$-axis position translations frame independent.
On all three axes, new velocity is calculated using acceleration, velocity is to player position, and then velocity is snapped.
However the key here is \emph{how} velocity is applied to the player position in regards to the $z$-axis, demonstrated in \eqref{eq:x_zk+1}.
\codeFromFile{firstline=63,lastline=69,gobble=1}{code/game/bg_slidemove.c}
On the $x$ and $y$ axes,
\begin{align*}
x_{x,k+1} = x_{x,k} + v_{x,k} T,\qquad x_{y,k+1} = x_{y,k} + v_{y,k} T.
\end{align*}
Translations on the $z$ axis are done as follows.
\begin{align*}
\text{Defining }&v_{z,k}\text{ as the sequence of $z$-axis velocities after snapping after $k$ frames}\\
\text{and }&x_{z,k}\text{ as the sequence of the player's $z$-axis position after $k$ frames},
\end{align*}
\begin{align}
\label{eq:v_zk+1}
v_{z,k+1} &= v_{z,k} - \round*{gT}\\
\label{eq:x_zk+1}
x_{z,k+1} &= x_{z,k} + \left(v_{z,k} - \frac{gT}{2} \right) T
\end{align}
\begin{align}
\nonumber
v_{z,1} &= v_{z,0} - \round*{gT}\\\nonumber
v_{z,2} &= v_{z,1} - \round*{gT}\\\nonumber
&= v_{z,0} - 2\round*{gT}\\
\label{eq:v_zk}
\therefore\ v_{z,k} &= v_{z,0} - k\round*{gT}.
\end{align}
Same approach is used to derive $x_{z,k}$, done in Appendix \ref{app:derive_x_zk}, yielding
\begin{align}
\label{eq:x_zk}
x_{z,k} &= x_{z,0} + k\left(v_{z,0} + \frac{\round*{gT} - gT}{2} \right) T - \frac{k^2}{2} \round*{gT} T\\\nonumber
&= x_{z,0} + kT\left(v_{z,0} + \frac{(1 - k)\round*{gT} - gT}{2} \right).
\end{align}

To solve for the frame $k_{peak}$ that you reach the peak, differentiate $x_{z,k}$.
\begin{align}
\nonumber
\frac{\diff x_{z,k}}{\diff k} &= \left(v_{z,0} + \frac{\round*{gT} - gT}{2} \right) T - k\round*{gT} T\\\nonumber
0 &= \left(v_{z,0} + \frac{\round*{gT} - gT}{2} \right) T - k\round*{gT} T\\\nonumber
\implies k &= \frac{v_{z,0} - \frac{gT}{2} + \frac{\round*{gT}}{2}}{\round*{gT}}\\\nonumber
&= \frac{v_{z,0} - \frac{gT}{2}}{\round*{gT}} + \frac{1}{2}\\\nonumber
\implies k_{peak} &= \round*{\frac{v_{z,0} - \frac{gT}{2}}{\round*{gT}} + \frac{1}{2}}\\
\therefore\ k_{peak} &= \ceil*{\frac{v_{z,0} - \frac{gT}{2}}{\round*{gT}}}
\end{align}
With $\qty{125}{fps}$ and jump velocity $v_{z,0} = \qty{270}{ups}$, the peak is reached after $k_{peak} = \qty{45}{frames}$.\\

The maximum height the player reaches in a jump is given by substituting $k_{peak}$ into $x_{z,k}$, with $x_{z,0} = 0$.
When step-up is introduced in Section \ref{sec:stepup} this formula changes.
\begin{align}
\label{eq:max_jump_height}
\ceil*{\frac{v_{z,0} - \frac{gT}{2}}{\round*{gT}}}\left(v_{z,0} + \frac{\round*{gT} - gT}{2} \right) T - \frac{1}{2}\ceil*{\frac{v_{z,0} - \frac{gT}{2}}{\round*{gT}}}^2 \round*{gT} T
\end{align}
\\
This implementation of $z$-axis translations is frame independent, as shown below.
\begin{align*}
v_{z,k} = v_{z,0} - k\round*{gT}\ \ldots\ \eqref{eq:v_zk}
\end{align*}
Using trapezoidal rule,
\begin{align*}
&\sum_{i=0}^{k-1} \left(\left(v_{z,0} - i\round*{gT} \right) + \left(v_{z,0} - (i + 1) \round*{gT} \right)\right) \frac{T}{2}\\
&\sum_{i=0}^{k-1} \left(v_{z,0} - \left(i + \frac{1}{2} \right) \round*{gT} \right) T\\
&\implies x_{z,k} = x_{z,0} + k v_{z,0} T - \frac{k^2}{2}\round*{gT} T,
\end{align*}
but $x_{z,k}$ is
\begin{align*}
x_{z,0} + k\left(v_{z,0} + \frac{\round*{gT} - gT}{2} \right) T - \frac{k^2}{2}\round*{gT} T\ \ldots\ \eqref{eq:x_zk}.
\end{align*}
Using inverse trapezoidal rule,
\begin{align}
\nonumber
&\sum_{i=0}^{k-1} \left(\left(v_{z,0} + \frac{\round*{gT} - gT}{2} \right) - \left(i + \frac{1}{2} \right) \round*{gT} \right) T\\\nonumber
&\sum_{i=0}^{k-1} \left(\left(\left(v_{z,0} + \frac{\round*{gT} - gT}{2} \right) - i\round*{gT} \right) + \left(\left(v_{z,0} + \frac{\round*{gT} - gT}{2} \right) - (i + 1)\round*{gT} \right) \right) \frac{T}{2}\\
\label{eq:corrected_v_zk}
&\implies v_{z,k} = \left(v_{z,0} + \frac{\round*{gT} - gT}{2} \right) - k\round*{gT}
\end{align}
Thus, $z$-axis position is updated using this ``corrected'' velocity, meaning $z$-axis translations are frame independent, since a fractional $k$ would still result in the correct change in position. Hence $k \in \symbb{R}^+$.\\

As $x_z$ is updated using ``corrected'' velocity \eqref{eq:corrected_v_zk} rather than \eqref{eq:v_zk}, the difference in applied velocity is
\begin{align*}
\eqref{eq:corrected_v_zk} - \eqref{eq:v_zk} &= v_{z,0} + \frac{\round*{gT} - gT}{2} - k\round*{gT} - v_{z,0} + k\round*{gT}\\
&= \frac{\round*{gT} - gT}{2}.
\end{align*}
%Hence, in terms of $x_z$ translations, there is an initial velocity of $v_{z,0} + \frac{\round*{gT} - gT}{2}$ rather than $v_{z,0}$.
This implementation results in a height ``error'' $h$. With $k_{gravity} \in \symbb{R}^+$ as the number of frames that gravity was applied, $h$ is given by
\begin{align}
\label{eq:height_error}
h = k_{gravity}T\frac{\round*{gT} - gT}{2}.
\end{align}
For example, a jump at $\qty{125}{fps}$ lasts $k_{gravity} = \qty{90}{frames}$, so the final $x_z$ of the player is $h = \qty{-0.144}{units}$ relative to the starting $x_z$\footnote{Since $h < 0$ the final position is below the starting position, assuming the player didn't clip into the ground plane.}.\\

As a side note, this implementation of $z$-axis translations is also frame rate independent $\Longleftrightarrow$ rounding is ignored, since
\begin{itemize}
\item
Vertical acceleration is linear (constant) and thus
\item
Ignoring rounding,
\begin{align*}
x_{z,k} &= x_{z,0} + k\left(v_{z,0} + \frac{gT - gT}{2} \right) T - \frac{k^2}{2} gT T\\
&= x_{z,0} + kv_{z,0} T - \frac{k^2 T^2 g}{2}\\
&= x_{z,0} + (kT) v_{z,0} - (kT)^2 \frac{g}{2},
\end{align*}
which is the same as the physical model equation.
Without ignoring rounding, the implementation is frame rate dependent, hence the differing jump heights by changing frame rate.
\end{itemize}

It can be shown that $T_p = 0.003$ results in the largest jump height, with effective-gravity of $666.\overline{6}$.
This is demonstrated in the graph below.
\begin{figure}[H]
	\centering
	\setlength\figureheight{4.8cm}
	\setlength\figurewidth{13cm}
	\includetikz{tikz/jump_height}
	\caption{Graph of jump height over $T$. This graph is only valid when $T$ is a multiple of $0.001$, as proper-T $T_p$ should be used on the axis. The range of possible jump heights is shown (\greenarea).}
	\label{fig:jump_height}
\end{figure}
A frame-time of $T_p = 0.003$ results in the largest possible jump height of $\qty{54.594}{units}$, while a frame-time of $T_p = 0.001$ results in the lowest jump height of $\qty{36.477}{units}$.
The normal condition of $T_p = 0.008$ is also displayed in the graph.\\
TODO: finish describing


\subsection{Step-up}
\label{sec:stepup}
TODO: explanation of step-up, special cases (if any)\\
The maximum height the player can step-up is $\qty{18}{units}$ (\texttt{STEPSIZE}).
Hence, to calculate the maximum height the player can jump including step-up, add $\qty{18}{units}$ to \eqref{eq:max_jump_height} %giving
%\begin{align}
%\label{eq:max_jump_height_stepup}
%\ceil*{\frac{v_{z,0} - \frac{gT}{2}}{\round*{gT}}}\left(v_{z,0} + \frac{\round*{gT} - gT}{2} \right) T - \frac{1}{2}\ceil*{\frac{v_{z,0} - \frac{gT}{2}}{\round*{gT}}}^2 \round*{gT} T + 18,
%\end{align}
which is $\qty{66.528}{units}$ at $\qty{125}{fps}$.

Provided that the player is at an appropriate height for a step-up to occur, in CPM the player will step-up if $v_z \le 370$.
However in VQ3, there are additional required conditions:
\begin{itemize}
\item
Either $v_z \le 0$ or
\item
$v_z > 0$ as long as the player is close enough to the plane below them (less than \texttt{STEPSIZE} units above), and the plane is sufficiently flat ($\rho \ge \ang{45.572996}$),
\end{itemize}
contrary to what the comments state in the code.
\codeFromFile{firstline=253,lastline=257,gobble=1}{code/game/bg_slidemove.c}


\subsection{Skimming}
\label{sec:skimming}
Skimming is a phenomenon in which the player is allowed a short window of time after jumping to continue travelling in a direction without losing speed upon colliding with a wall, instead just gliding or ``skimming'' past it.
This is a result of the game engine only applying gravity, allowing acceleration, but not clipping velocity whenever the ``skim timer'' (\texttt{pm->ps->pm\_time}) is active.
TODO: what the skimming code intended to achieve\\
Note: \texttt{primal\_velocity} here would be a copy of \texttt{pm->ps->velocity} but with gravity applied.
\codeFromFile{firstline=219,lastline=222,gobble=1}{code/game/bg_slidemove.c}
The main method in order to start the skim timer is landing on a ground plane after jumpnig with previous $z$ velocity $< -200$.
This sets the skim timer to $\qty{250}{ms}$.\\
Note that skimming can still occur after a step-up rather than a normal landing, due to the code making use of the previous velocity before it is clipped to $0$.
\codeFromFile{firstline=1184,lastline=1188,gobble=2}{code/game/bg_pmove.c}
Another method to start the skim timer is by performing a water jump. TODO: how.
This sets the skim timer to $\qty{2000}{ms}$.
\codeFromFile{firstline=432,lastline=437,gobble=1}{code/game/bg_pmove.c}
The skim timer can also be started by taking damage from something that induces knockback, for example weapons.
This sets the skim timer to $\min(200, \max(50, \texttt{damage}\times2))$ ms. Here \texttt{knockback} is set to $\texttt{damage}\times2$.
\codeFromFile{firstline=920,lastline=930,gobble=2}{code/game/g_combat.c}
Finally, the skim timer can be started by going through a teleporter.
This sets the skim timer to $\qty{160}{ms}$.
\codeFromFile{firstline=101,lastline=101,gobble=1}{code/game/g_misc.c}


\subsection{Overbounces}
\label{sec:overbounce}
TODO%\\
%Overbounces occur when $x_z$ comes within 0 and 0.25 units of a ground plane.


\subsection{Wallbug}
\label{sec:wallbug}
Wallbugs occur due to $z$-axis velocity being zeroed when the game thinks the player entity is inside a solid when they actually aren't.
Zeroing the $z$-axis velocity causes the player to stop falling, but still be able to accelerate by strafing along the $xy$-plane.
\codeFromFile{firstline=99,lastline=103,gobble=2}{code/game/bg_slidemove.c}
TODO: escape conditions
