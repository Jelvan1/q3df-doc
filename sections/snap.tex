\section{SnapHUD}
\label{sec:snaphud}

\subsection{Basics}
\label{sec:snap_basics}
In order to save resources, the game engine rounds each component of some vectors each frame, as shown by the below code. Albeit a simple change, this largely changes the way the game is played.
\codeFromFile{firstline=151,lastline=156}{code/unix/unix_shared.c}
Since player velocity is one of the vectors that is rounded, we can interpret this as rounding the speed change the player receives each frame, forming
\begin{align*}
\round{\vec{a}} &=
\begin{pmatrix}
\round{a_x} \\ \round{a_y} \\ \round{a_z}
\end{pmatrix}, & \round{\vec{r}} &=
\begin{pmatrix}
\round{r_x} \\ \round{r_y} \\ \round{r_z}
\end{pmatrix}.
\end{align*}
Hence, there are specific constant view yaw angle ranges called ``snap zones'' which provide the same acceleration regardless of where the player is looking in the zone.
Some of these zones are rounded to give a value $\norm*{\round*{\vec{\flat{a}}}} > sAT$, and similarly some are rounded such that $\norm*{\round*{\vec{\flat{a}}}} < sAT$.
Note that the player will still only accelerate if $\delta \in [\delta_{\min}, 2\pi - \delta_{\min}]$, using the new value of $\delta_{\min}$ after taking into account velocity snapping.

Additionally, the closer $\delta_{\min}$ is to the border of a snap zone, there will be a greater increase in speed, as the acceleration is pointed more in the direction of $\vec{\flat{v}}$ meaning more of the acceleration goes into increasing the speed rather than changing the direction.
This also means that as $\delta_{\min}$ gets closer to the border of a snap zone, $\delta_{\min}$ will change faster due to speed increasing more rapidly.
Players often attempt to optimize (by decreasing) the distance between $\delta_{\min}$ and the border of the snap zone in order to maximize acceleration through techniques referred to as ``snap manipulation'' or ``snap zone manipulation.''\\

Furthermore, due to velocity snapping rotating the acceleration vector, given a sufficiently high velocity it is possible to lose speed while strafing at $\delta_{\opt}$, which is discussed in Section \ref{sec:speed_loss_while_strafing}.\\

We can form an illustration of the snap zones in any conditions given that we know the acceleration the player is receiving in that frame.
In normal conditions while air strafing at $\qty{125}{fps}$ with no moveup and $\delta \in [\delta_{\opt}, 2\pi - \delta_{\opt}]$, $\flat{a} = 2.56$. The effect of the rounding can be visualized with a circle with radius 2.56.\\

TODO


\subsection{Frame rate impact}
\label{sec:snap_frame_rate}
TODO


\subsection{Proper maximum ground speed}
\label{sec:max_ground_speed}
The addition of velocity snapping also affects the maximum ground speed previously calculated to be $\flat{v}_{\max}$.
There are technically two solutions to the proper maximum ground speed for CPM and VQ3: allowing the player to turn any direction (clockwise and anticlockwise), or restricting the player's rotation to one direction as is done in a circle jump.\\
TODO: method\\

Results\\
\begin{tabular*}{\textwidth}{c @{\extracolsep{\fill}} cc}
Restricted && Unrestricted\\
& CPM results &\\
$\flat{v}_{\max\_\proper} = \norm*{\begin{pmatrix}
TODO \\ TODO \\ 0
\end{pmatrix}} = \qty{498.787}{ups}$ && $\flat{v}_{\max\_\proper} = \norm*{\begin{pmatrix}
TODO \\ TODO \\ 0
\end{pmatrix}} = \qty{500.3}{ups}$\\
& VQ3 results &\\
$\flat{v}_{\max\_\proper} = \norm*{\begin{pmatrix}
TODO \\ TODO \\ 0
\end{pmatrix}} = TODO$ && $\flat{v}_{\max\_\proper} = \norm*{\begin{pmatrix}
403 \\ 91 \\ 0
\end{pmatrix}} = \qty{413.146}{ups}$
\end{tabular*}


\subsection{Speed loss while strafing optimally}
\label{sec:speed_loss_while_strafing}
As mentioned before, given a high enough velocity the player can lose speed while strafing with $\delta \in [\delta_{\opt}, 2\pi - \delta_{\opt}]$. CPM players often attempt to counteract this through snap zone manipulation, by W-turning through such areas of speed loss.\\

While air strafing, the lowest velocity that such speed loss can occur can be calculated. First it would be useful to define \emph{symmetrical snap zones} as snap zones where the respective $\round*{\vec{\flat{a}}}$ vector lies in the middle of the snap zone borders. It does not matter which direction these zones are approached from. Similarly, \emph{asymmetrical snap zones} are snap zones where this is not the case.
There is 1 symmetrical snap zone when $\flat{a} < 0.5$ and 8 when $\flat{a}\ge 0.5$, logically being the snap zones along the axes and diagonals. The first asymmetrical snap zones appear after $\flat{a}\ge 1.5$.
A loss in speed from strafing can first occur in a zone with the largest difference (in terms of angle) between a border of such snap zone and the respective $\round*{\vec{\flat{a}}}$ for that zone.\\

In normal conditions of $\qty{125}{fps}$ while air strafing with no upmove, this difference $\Delta$ is equal to
\begin{align*}
\atan\left(\frac{1}{2} \right) - \atan\left(\frac{\sqrt{\flat{a}^2 - 2.5^2}}{2.5} \right) = \qty{0.246716}{rad},
\end{align*}
and occurs when entering the big uncoloured snap zone from the small zone,\footnote{TODO: explain each of the 24 snap zones} and hence $\norm*{\round*{\vec{\flat{a}}}} = \sqrt{1^2 + 2^2} = \sqrt{5}$. Since velocity snapping changes the received acceleration and a proper $\delta_{\max}$ value depends on the acceleration after velocity snapping, we must first define $\delta_{\max\_\proper}$ as
\begin{align}
\delta_{\max\_\proper} &= \acos\left(\frac{\flat{v}^2 - \flat{v}_f^2 - \norm*{\round*{\vec{\flat{a}}}}^2}{2\norm*{\round*{\vec{\flat{a}}}}\flat{v}_f} \right)\\\nonumber
&= \acos\left(-\frac{\norm*{\round*{\vec{\flat{a}}}}}{2\flat{v}} \right)\ \ldots\ \text{while in air.}
\end{align}
Note that definitions of all other proper $\delta$ variables are situation-based so a formula similar to this cannot realistically be derived.
Assuming the player is strafing at $\delta_{\opt}$, velocities at which speed can be lost by strafing can hence be calculated by solving
\begin{align*}
\delta_{\opt} + \Delta &> \delta_{\max\_\proper}\\
\acos\left(\frac{s - \flat{a}}{\flat{v}} \right) + \Delta &> \acos\left(-\frac{\norm*{\round*{\vec{\flat{a}}}}}{2\flat{v}} \right),
\end{align*}
giving $\flat{v} > \qty{1304.247}{ups}$, which first occurs with
\begin{align*}
\vec{v} =
\begin{pmatrix}
583 \\ -1169 \\ v_z
\end{pmatrix}.
\end{align*}
It should be noted this solution is frame rate dependent, and that the solution can be reflected about the $x$ and $y$ axes and diagonals to produce 7 other equivalent solutions.
