\title{Quake III DeFRaG Physics}
\author{}
\date{\today}

\documentclass[10pt]{article}

\usepackage[a4paper,margin=27.5mm]{geometry}
\usepackage[english]{babel}
\usepackage{appendix}
\usepackage[nottoc]{tocbibind} % add bibliography to toc

\usepackage{pgfplots}
\pgfplotsset{compat=1.18} % latest version on 2021-12-25
\usepackage{mathtools} % updates and extends amsmath
\usepackage{amssymb} % mathbb
\usepackage{siunitx} % units
\usepackage{subcaption} % subfigure
\usepackage[normalem]{ulem} % uline

\usepackage[pdfusetitle,colorlinks=true]{hyperref} % recommended to load last
\usepackage[all]{tcolorbox} % listing (requires hyperref)

\usepgfplotslibrary{external}
\usetikzlibrary{
	3d,                        % plot 3D shapes
	arrows.meta,               % arrow tip library
	calc,                      % allows advanced coordinate calculations
	decorations.pathreplacing, % replace to-be-decorated path by another path
	patterns,                  % patterns for filling areas
	spy,                       % magnified area
}
\tikzexternalize
\newcommand*{\includetikz}[1]{
	\tikzsetnextfilename{#1} % changes filename of next externalized figure
	\input{#1.tikz}
}
\pgfdeclarepatternformonly{north east lines wide}{\pgfqpoint{-1pt}{-1pt}}{\pgfqpoint{10pt}{10pt}}{\pgfqpoint{9pt}{9pt}}%
{
	\pgfsetlinewidth{0.4pt}
	\pgfpathmoveto{\pgfqpoint{0pt}{0pt}}
	\pgfpathlineto{\pgfqpoint{9.1pt}{9.1pt}}
	\pgfusepath{stroke}
}
\tikzset{
	body/.style={inner sep=0pt,outer sep=0pt,shape=rectangle,draw,thick,pattern=north east lines wide},
	dimen/.style={<->,>=latex,thin,every rectangle node/.style={fill=white,midway}},
	symmetry/.style={dashed,thin},
}

%%%%%%%%%%%%%%%%%%%%%%%%%%%%%%%%%%%%%%%%%%%%%%%%%%%%%%%

\newcommand*{\inserteqstrut}[1]{%
	\rlap{$\displaystyle#1$}%
	\phantom{\biggesteq}}
\newcommand*{\inserteqstrutt}[1]{%
	\rlap{$\displaystyle#1$}%
	\phantom{\biggesteqq}}

\newlength{\letterwidth}
\newcommand*{\mathmakeboxlargestof}[3]{%
	% #1 = first symbol
	% #2 = second symbol
	% #3 = what is printed
	\setlength{\letterwidth}{\maxof{\widthof{$#1$}}{\widthof{$#2$}}}%
	\mathmakebox[\letterwidth]{\vphantom{#1}\vphantom{#2}#3}%
}

\newcommand*{\lcr}[3]{%
\mathmakebox[\textwidth][s]{\mathrlap{#1}\hfill#2\hfill\mathllap{#3}}%
}

% vertical centering of inline tikz with [baseline=-\the\dimexpr\fontdimen22\textfont2\relax]
% https://tex.stackexchange.com/questions/59658/use-of-tikzpicture-matrix-in-align-or-gather-environment/59660#comment126261_59660
\DeclareRobustCommand{\blueline}{\tikzexternaldisable\tikz[baseline=-\the\dimexpr\fontdimen22\textfont2\relax]{\draw[mycolor1,solid,line width=1.0pt](0,0) -- (5mm,0);}\tikzexternalenable}
\DeclareRobustCommand{\bluedash}{\tikzexternaldisable\tikz[baseline=-\the\dimexpr\fontdimen22\textfont2\relax]{\draw[mycolor1,dashed,line width=1.0pt](0,0) -- (5mm,0);}\tikzexternalenable}
\DeclareRobustCommand{\orangeline}{\tikzexternaldisable\tikz[baseline=-\the\dimexpr\fontdimen22\textfont2\relax]{\draw[mycolor2,solid,line width=1.0pt](0,0) -- (5mm,0);}\tikzexternalenable}
\DeclareRobustCommand{\greyline}{\tikzexternaldisable\tikz[baseline=-\the\dimexpr\fontdimen22\textfont2\relax]{\draw[mycolor0,solid,line width=1.0pt](0,0) -- (5mm,0);}\tikzexternalenable}
\DeclareRobustCommand{\greenline}{\tikzexternaldisable\tikz[baseline=-\the\dimexpr\fontdimen22\textfont2\relax]{\draw[mycolor5,solid,line width=1.0pt](0,0) -- (5mm,0);}\tikzexternalenable}
\DeclareRobustCommand{\blackline}{\tikzexternaldisable\tikz[baseline=-\the\dimexpr\fontdimen22\textfont2\relax]{\draw[black,solid,line width=1.0pt](0,0) -- (5mm,0);}\tikzexternalenable}

\DeclareRobustCommand{\thickblueline}{\tikzexternaldisable\tikz[baseline=-\the\dimexpr\fontdimen22\textfont2\relax]{\draw[mycolor1,solid,line width=2.0pt](0,0) -- (5mm,0);}\tikzexternalenable}
\DeclareRobustCommand{\thickCPMorangeline}{\tikzexternaldisable\tikz[baseline=-\the\dimexpr\fontdimen22\textfont2\relax]{\draw[mycolor2,solid,line width=2.0pt](0,0) -- (5mm,0);}\tikzexternalenable}

\DeclareRobustCommand{\greydash}{\tikzexternaldisable\tikz[baseline=-\the\dimexpr\fontdimen22\textfont2\relax]{\draw[mycolor0,dashed,line width=1.0pt](0,0) -- (5mm,0);}\tikzexternalenable}
\DeclareRobustCommand{\yellowdenselydottedarrow}{\tikzexternaldisable\tikz[baseline=-\the\dimexpr\fontdimen22\textfont2\relax]{\draw[->,mycolor3,densely dotted,very thick](0,0) -- (5mm,0);}\tikzexternalenable}

\DeclareRobustCommand{\bluearrow}{\tikzexternaldisable\tikz[baseline=-\the\dimexpr\fontdimen22\textfont2\relax]{\draw[->,mycolor1,solid,very thick](0,0) -- (5mm,0);}\tikzexternalenable}
\DeclareRobustCommand{\orangearrow}{\tikzexternaldisable\tikz[baseline=-\the\dimexpr\fontdimen22\textfont2\relax]{\draw[->,mycolor2,solid,very thick](0,0) -- (5mm,0);}\tikzexternalenable}
\DeclareRobustCommand{\yellowarrow}{\tikzexternaldisable\tikz[baseline=-\the\dimexpr\fontdimen22\textfont2\relax]{\draw[->,mycolor3,solid,very thick](0,0) -- (5mm,0);}\tikzexternalenable}
\DeclareRobustCommand{\purplearrow}{\tikzexternaldisable\tikz[baseline=-\the\dimexpr\fontdimen22\textfont2\relax]{\draw[->,mycolor4,solid,very thick](0,0) -- (5mm,0);}\tikzexternalenable}
\DeclareRobustCommand{\greenarrow}{\tikzexternaldisable\tikz[baseline=-\the\dimexpr\fontdimen22\textfont2\relax]{\draw[->,mycolor5,solid,very thick](0,0) -- (5mm,0);}\tikzexternalenable}

\DeclareRobustCommand{\greyarea}{\tikzexternaldisable\tikz[baseline=-\the\dimexpr\fontdimen22\textfont2\relax+1.25mm]{\draw[fill=mycolor0!10,draw=mycolor0] (0mm, 0mm) rectangle (5mm, 2.5mm);}\tikzexternalenable}
\DeclareRobustCommand{\bluearea}{\tikzexternaldisable\tikz[baseline=-\the\dimexpr\fontdimen22\textfont2\relax+1.25mm]{\draw[fill=mycolor1!40,draw=black!40!mycolor1] (0mm, 0mm) rectangle (5mm, 2.5mm);}\tikzexternalenable}
\DeclareRobustCommand{\orangearea}{\tikzexternaldisable\tikz[baseline=-\the\dimexpr\fontdimen22\textfont2\relax+1.25mm]{\draw[fill=mycolor2!40,draw=black!40!mycolor2] (0mm, 0mm) rectangle (5mm, 2.5mm);}\tikzexternalenable}
\DeclareRobustCommand{\greenarea}{\tikzexternaldisable\tikz[baseline=-\the\dimexpr\fontdimen22\textfont2\relax+1.25mm]{\draw[fill=mycolor5!40,draw=black!40!mycolor5] (0mm, 0mm) rectangle (5mm, 2.5mm);}\tikzexternalenable}

\DeclareRobustCommand{\darkbluearea}{\tikzexternaldisable\tikz[baseline=-\the\dimexpr\fontdimen22\textfont2\relax+1.25mm]{\draw[fill=mycolor1,draw=black] (0mm, 0mm) rectangle (5mm, 2.5mm);}\tikzexternalenable}
\DeclareRobustCommand{\lightorangearea}{\tikzexternaldisable\tikz[baseline=-\the\dimexpr\fontdimen22\textfont2\relax+1.25mm]{\draw[fill=mycolor2!10,draw=mycolor2] (0mm, 0mm) rectangle (5mm, 2.5mm);}\tikzexternalenable}
\DeclareRobustCommand{\darkorangearea}{\tikzexternaldisable\tikz[baseline=-\the\dimexpr\fontdimen22\textfont2\relax+1.25mm]{\draw[fill=mycolor2,draw=black] (0mm, 0mm) rectangle (5mm, 2.5mm);}\tikzexternalenable}
\DeclareRobustCommand{\darkgreyarea}{\tikzexternaldisable\tikz[baseline=-\the\dimexpr\fontdimen22\textfont2\relax+1.25mm]{\draw[fill=mycolor0!60,draw=mycolor0] (0mm, 0mm) rectangle (5mm, 2.5mm);}\tikzexternalenable}

\newcommand*{\abs}[1]{\left\lvert#1\right\rvert}
\newcommand*{\norm}[1]{\left\lVert#1\right\rVert}
\newcommand*{\set}[1]{\left\{#1\right\}}
\renewcommand*{\vec}[1]{\boldsymbol{#1}}
\newcommand*{\uvec}[1]{\vec{\hat{#1}}}
\newcommand*{\mat}[1]{\boldsymbol{#1}}
\newcommand*{\projmat}[1]{\mat{P}_{\vec{#1}}}
\newcommand*{\resprojmat}[1]{\mat{M}_{\vec{#1}}}
\makeatletter
\newcommand*{\inlinemat}[2][~]{%
	\left(%
	\def\nextitem{\def\nextitem{#1}}%
	\@for \el:=#2\do{\nextitem\el}%
	\right)%
}
\makeatother
\renewcommand*{\flat}[1]{\uline{#1}}

\DeclareMathOperator{\acos}{acos}
\DeclareMathOperator{\atan}{atan}
\DeclareMathOperator{\opt}{opt}
\DeclareMathOperator{\yaw}{yaw}
\DeclareMathOperator{\cmd}{cmd}
\DeclareMathOperator{\fmove}{fm}
\DeclareMathOperator{\rmove}{rm}
\DeclareMathOperator{\umove}{um}
\DeclareMathOperator{\fumove}{fum}

\newlength\figureheight
\newlength\figurewidth

\definecolor{mycolor0}{rgb}{.4,.4,.4}%
\definecolor{mycolor1}{rgb}{0.00000,0.44700,0.74100}%
\definecolor{mycolor2}{rgb}{0.85000,0.32500,0.09800}%
\definecolor{mycolor3}{rgb}{0.92900,0.69400,0.12500}%
\definecolor{mycolor4}{rgb}{0.49400,0.18400,0.55600}%
\definecolor{mycolor5}{rgb}{0.46600,0.67400,0.18800}%

\definecolor{cgazgreen}{rgb}{0,1,0}%
\definecolor{cgazdarkgreen}{rgb}{0,0.25,0.25}%
\definecolor{cgazyellow}{rgb}{1,1,0}%
\definecolor{cgazgrey}{rgb}{0.25,0.25,0.25}%

\newcommand*{\highlightcgazgreen}[1]{\colorbox{cgazgreen!50}{$\displaystyle#1$}}
\newcommand*{\highlightcgazdarkgreen}[1]{\colorbox{cgazdarkgreen!50}{$\displaystyle#1$}}
\newcommand*{\highlightcgazyellow}[1]{\colorbox{cgazyellow!50}{$\displaystyle#1$}}
\newcommand*{\highlightcgazgrey}[1]{\colorbox{cgazgrey!50}{$\displaystyle#1$}}

\definecolor{bgcolor}{rgb}{0.047058824,0.047058824,0.058823529}
\definecolor{commentcolor}{rgb}{0.364705882,0.352941176,0.4}
\definecolor{fgcolor}{rgb}{0.541176471,0.541176471,0.576470588}
\definecolor{identifiercolor}{rgb}{0.266666667,0.635294118,0.862745098}
\definecolor{keywordcolor}{rgb}{0.764705882,0.231372549,0.270588235}
\definecolor{stringcolor}{rgb}{0.792156863,0.623529412,0.37254902}

% https://tex.stackexchange.com/questions/20890/define-an-escape-underscore-environment
\makeatletter
\DeclareRobustCommand*{\escapeus}[1]{%
	\begingroup\@activeus\scantokens{#1\endinput}\endgroup}
\begingroup\lccode`\~=`\_\relax
\lowercase{\endgroup\def\@activeus{\catcode`\_=\active \let~\_}}
\makeatother

% https://tex.stackexchange.com/questions/486102/problems-with-tcolorbox-and-tikz-external
%\tcbset{shield externalize}
\tcbsetforeverylayer{shield externalize}% <--- interim solution before bug fix
\renewcommand{\theFancyVerbLine}{\ttfamily \textcolor[rgb]{0,0,0}{\scriptsize \arabic{FancyVerbLine}}}
\newtcbinputlisting[]{\codeFromFile}[3][]{%
	listing engine=minted,
	listing file={#3},
	listing only,
	listing remove caption=false,
	minted language=C,
	colframe=mycolor0!30,
	colback=mycolor0!10,
	borderline west={8mm}{-4mm}{mycolor0!30},
	breakable,
	enhanced,
	top=0mm,
	bottom=0mm,
	before skip=.75\baselineskip,
	after skip=.75\baselineskip,
	attach boxed title to bottom center={yshift=-.75\baselineskip},
	minipage boxed title,
	boxed title style={blanker},
	title={\captionof{listing}{Code snippet from \href{https://github.com/id-Software/Quake-III-Arena/blob/master/#3}{\texttt{\escapeus{#3}}}.}}, % TODO: URL with line range
	minted options={
		%frame=single,
		%rulecolor=\color{mycolor0!10},
		fontsize=\footnotesize,
		linenos=true,
		stripall=true,
		numbersep=2mm,
		#2},
	arc=0pt,
	outer arc=0pt,
	boxrule=1pt,
	#1}

\begin{document}
\maketitle

\tableofcontents

\section{Fundamentals}
\label{sec:fundamentals}
\cite{hl_physics}

\subsection{Input}
% Appendix \ref{app:angle_vectors}

\subsection{Friction}
Before any acceleration comes into play, friction is applied first. The amount of friction depends on the player's speed $v$, the friction factor $c$%
% TODO: don't use footnotes for c, T and d)
\footnote{On the ground the friction factor $c=6$ (\texttt{pm\_friction}), but in the air $c=0$.} and the frame-time $T$\footnote{The frame-time $T = 1 / \qty{125}{fps} = \qty{0.008}{s}$.}. At lower speeds, the stopspeed limit $d$\footnote{The stopspeed limit $d = \qty{100}{ups}$ (\texttt{pm\_stopspeed}).} kicks in and brings the player to a complete stop faster. The current velocity vector before and after friction is applied can then be defined as respectively
\begin{align*}
\vec{v} =
\begin{pmatrix}
v_x \\ v_y \\ v_z
\end{pmatrix} = v \uvec{v},
&&
\vec{v}_f =
\begin{pmatrix}
v_{fx} \\ v_{fy} \\ v_{fz}
\end{pmatrix} = v_f \uvec{v},
\end{align*}
with the same direction $\uvec{v}$ and the magnitude relation
\begin{align}
\label{eq:vf}
v_f = (1 - \iota cT) v,
\end{align}
assuming $cT \le 1$ and where
\begin{align*}
\iota = \max\left(1, \min\left(\frac{d}{\flat{v}}, \frac{1}{cT}\right)\right)\protect\footnotemark{}.
\end{align*}
\footnotetext{This is a simple clamping operation.}%
This means that, without acceleration, the current velocity either remains unchanged or decreases over time ($0 \le v_f \le v$).

% Appendix \ref{app:friction}

\subsection{Acceleration}
This would not be an interesting game without the ability to increase speed and change direction. So let's define the acceleration vector as
\begin{align*}
\vec{a} &=
\begin{pmatrix}
a_x \\ a_y \\ a_z
\end{pmatrix} = a \uvec{a},
\end{align*}
and the resulting new velocity becomes
\begin{align*}
\vec{r} &= \vec{v}_{f} + \vec{a} =
\begin{pmatrix}
r_x \\ r_y \\ r_z
\end{pmatrix} = r \uvec{r}.
\end{align*}
The relation between all defined vectors is summarized in Figure \ref{fig:delta_phi}, where the normal of the (horizontal) $xy$-plane (\greenarea) is defined as
\begin{align*}
\vec{\flat{n}} &=
\begin{pmatrix}
0\\0\\1
\end{pmatrix}.
\end{align*}
\begin{figure}[H]
	\centering
	\begin{subfigure}[t]{.5\textwidth}
		\centering
		\setlength\figureheight{5.5cm}
		\setlength\figurewidth{5.5cm}
		\includetikz{tikz/delta_phi}
		\caption{}
	\end{subfigure}%
	\begin{subfigure}[t]{.5\textwidth}
		\centering
		\setlength\figureheight{9.5cm}
		\setlength\figurewidth{9.5cm}
		\includetikz{tikz/delta_phi3d}
		\caption{}
	\end{subfigure}%
	\caption{The surface normal $\vec{\flat{n}}$ (\greenarrow), the current velocity $\vec{v}_f$ (\yellowdenselydottedarrow) and its projection $\vec{\flat{v}}_f$ (\yellowarrow), the projected acceleration $\vec{\flat{a}}$ (\orangearrow) and the resulting new velocity $\vec{\flat{r}}$ (\bluearrow). (a) is the top view of (b).}
	\label{fig:delta_phi}
\end{figure}

The matrix that projects all vectors orthogonally onto the $xy$-plane (\greenarea) is then
\begin{align*}
\resprojmat{\flat{n}} = \mat{I} - \vec{\flat{n}} \vec{\flat{n}}^T &=
\begin{pmatrix}
1 & 0 & 0\\
0 & 1 & 0\\
0 & 0 & 0
\end{pmatrix}
\end{align*}
Hence, the current velocity vectors projected to the $xy$-plane become
\begin{align*}
\vec{\flat{v}} &= \resprojmat{\flat{n}} \vec{v}\hphantom{_f} =
\begin{pmatrix}
\mathmakebox[\widthof{$v_{fx}$}]{v_x} \\ v_y \\0
\end{pmatrix} = \mathmakebox[\widthof{$\flat{v}_f$}]{\flat{v}} \uvec{\flat{v}},\\
\vec{\flat{v}}_f &= \resprojmat{\flat{n}} \vec{v}_f =
\begin{pmatrix}
v_{fx} \\ v_{fy} \\0
\end{pmatrix} = \flat{v}_f \uvec{\flat{v}}
\end{align*}
and the projected equivalent of equation \eqref{eq:vf} is
\begin{align}
\label{eq:flat_vf}
\flat{v}_f = (1 - \iota cT) \flat{v}.
\end{align}
Likewise, the projected acceleration vector is
\begin{align}
\label{eq:accel_direction}
\vec{\flat{a}} = \resprojmat{\flat{n}} \vec{a} =
\begin{pmatrix}
a_x \\ a_y \\0
\end{pmatrix} = \flat{a} \uvec{\flat{a}},
\end{align}
with a magnitude
\begin{align}
\label{eq:sAT}
\flat{a} = sAT,
\end{align}
where $A$ is some dimensionless constant depending on the player state\footnote{In the air, $A = 1$ (\texttt{pm\_airaccelerate}), while on the ground $A = 10$ and $A = 15$ (\texttt{pm\_accelerate}) for VQ3 and CPM, respectively.} and $s$ is the wishspeed limit (e.g. $\texttt{g\_speed} = \qty{320}{ups}$ on a flat surface under normal conditions, see more details in Section \ref{sec:movementkeys}). In the same way, the projection of the resulting new velocity becomes
\begin{align*}
\vec{\flat{r}} = \resprojmat{\flat{n}} \vec{r} =
\begin{pmatrix}
r_x \\ r_y \\0
\end{pmatrix} = \flat{r} \uvec{\flat{r}}.
\end{align*}

Thus, to project a vector onto the current velocity vector $\vec{\flat{v}}$ lying on the surface (\greenarea), it becomes
\begin{alignat*}{4}
\projmat{\flat{v}} &= \uvec{\flat{v}} \uvec{\flat{v}}^T &&= \frac{\vec{\flat{v}} \vec{\flat{v}}^T}{\vec{\flat{v}}^T \vec{\flat{v}}} &&= \frac{1}{v_x^2 + v_y^2}
\begin{pmatrix}
v_x^2 & v_x v_y & 0\\
v_x v_y & v_y^2 & 0\\
0 & 0 & 0\\
\end{pmatrix},\\
%&&&= \frac{\vec{\flat{v}}_f \vec{\flat{v}}_f^T}{\vec{\flat{v}}_f^T \vec{\flat{v}}_f} &&= \frac{\resprojmat{\flat{n}}\vec{v}_f\vec{v}_f^T\resprojmat{\flat{n}}^T}{\vec{v}_f^T\resprojmat{\flat{n}}\vec{v}_f} &&= \projmat{\flat{v}_f}^T = \projmat{\flat{v}_f}^k, \qquad k \in \symbb{N}^+,\\
%
\norm{\projmat{\flat{v}} \vec{\flat{r}}} &= \mathmakebox[0pt][l]{\sqrt{\vec{\flat{r}}^T \projmat{\flat{v}} \vec{\flat{r}}} = \sqrt{\vec{\flat{r}}^T \uvec{\flat{v}} \uvec{\flat{v}}^T \vec{\flat{r}}} = \vec{\flat{r}}^T \uvec{\flat{v}} = \flat{r} \cos\varphi,}
\end{alignat*}
where $\varphi$ represents the angle between the current velocity $\vec{\flat{v}}$ and the new velocity $\vec{\flat{r}}$. Likewise, to project a vector onto the acceleration vector $\vec{\flat{a}}$, we need to multiply with
\begin{align*}
\projmat{\flat{a}} &= \uvec{\flat{a}} \uvec{\flat{a}}^T = \frac{\vec{\flat{a}} \vec{\flat{a}}^T}{\vec{\flat{a}}^T \vec{\flat{a}}} =
%\begin{pmatrix}
%\cos^2\delta & \cos\delta\sin\delta & 0\\
%\cos\delta\sin\delta & \sin^2\delta & 0\\
%0 & 0 & 0
%\end{pmatrix} =
\projmat{\flat{a}}^T = \projmat{\flat{a}}^k, \qquad k \in \symbb{N}^+,\\
%
\norm{\projmat{\flat{a}} \vec{\flat{v}}_f} &= \sqrt{\vec{\flat{v}}_f^T \projmat{\flat{a}} \vec{\flat{v}}_f} = \sqrt{\vec{\flat{v}}_f^T \uvec{\flat{a}} \uvec{\flat{a}}^T \vec{\flat{v}}_f} = \vec{\flat{v}}_f^T \uvec{\flat{a}} = \flat{v}_f \cos\delta,
\end{align*}
where $\delta$ represents the angle between the acceleration $\vec{\flat{a}}$ and the velocity $\vec{\flat{v}}$.

\section{CGazHUD}
\label{sec:cgazhud}
\cite{injx_physics}\\
Short for CampingGaz-HUD.

\subsection{Strafing on flat ground or in air}
\label{sec:normal_strafe}
When we try to apply an acceleration $\vec{\flat{a}}$ at some angle $\delta$ to the velocity $\vec{\flat{v}}_f$, we will be allowed to accelerate provided that the component of our current velocity in the proposed direction is smaller than $s$, also expressed as $\flat{v}_f \cos\delta < s$. The size of the acceleration applied, $\flat{a}$, is a constant\footnote{In the air $\flat{a} = 2.56$, while on the ground $\flat{a} = 25.6$ and $\flat{a} = 38.4$ for VQ3 and CPM, respectively.}, independent of the angle at which it is applied. However, there is one special case which is an exception to this rule. If $\flat{v}_f \cos\delta < s$ but still very close to $s$, applying the usual constant acceleration $\flat{a}$ for a single frame would be enough to take you over $s$ in that direction. This is something the engine checks for, and if the situation exists, the applied acceleration will be only enough to take you up to $s$ (so the resulting acceleration would be $s - \flat{v}_f \cos\delta$ instead of the full acceleration $\flat{a}$). However, the range of angle $\delta$ at which this situation exists is very small ($\delta_{\min} \le \delta \le \delta_{\opt}$) as can be seen in Figure \ref{fig:delta_min_friction}--\ref{fig:delta_opt_friction}.
\begin{figure}[H]
	\centering
	\begin{subfigure}[t]{.33\textwidth}
		\centering
		\setlength\figureheight{5.5cm}
		\setlength\figurewidth{5.5cm}
		\includetikz{tikz/delta_min_friction}
		\caption{}
		\label{fig:delta_min_friction}
	\end{subfigure}%
	\begin{subfigure}[t]{.33\textwidth}
		\centering
		\setlength\figureheight{5.5cm}
		\setlength\figurewidth{5.5cm}
		\includetikz{tikz/delta_opt_friction}
		\caption{}
		\label{fig:delta_opt_friction}
	\end{subfigure}%
	\begin{subfigure}[t]{.33\textwidth}
		\centering
		\setlength\figureheight{5.5cm}
		\setlength\figurewidth{5.5cm}
		\includetikz{tikz/delta_max_friction}
		\caption{}
		\label{fig:delta_max_friction}
	\end{subfigure}
	\caption{The current velocity $\vec{\flat{v}}$ (\yellowdenselydottedarrow) and the velocity after friction $\vec{\flat{v}}_f$ (\yellowarrow) together with all possible acceleration configurations (\lightorangearea).}
	%	\label{fig:delta}
\end{figure}
Hence, we can summarize that
\begin{align}
\label{eq:r}
\vec{\flat{r}} &=
\begin{cases}
\vec{\flat{v}}_f, & s - \flat{v}_f \cos\delta \le 0,\\
\vec{\flat{v}}_f + (s - \flat{v}_f \cos\delta)\uvec{\flat{a}}, & 0 < s - \flat{v}_f \cos\delta \le \flat{a},\\
\vec{\flat{v}}_f + \mathmakebox[\widthof{$(s - \flat{v}_f \cos\delta)$}][r]{\flat{a}} \uvec{\flat{a}}, & s - \flat{v}_f \cos\delta > \flat{a}.
\end{cases}
\end{align}

The acceleration causes a change in the total velocity, which is made up of a \emph{change in speed} and a \emph{change in direction}. The direction $\uvec{a}$ in which the acceleration is applied dictates how much of this acceleration goes into changing the speed, and how much goes into changing the direction. In \emph{strafe-jumping}, it is the \emph{change in speed} we are interested in \emph{maximizing}, in an increasing fashion.

The way to achieve this maximum increase in speed is by applying the acceleration so its direction is as close to that of the original velocity as possible, i.e. so the angle $\delta$ is as small as possible. This will mean that more of the acceleration goes into increasing our speed, and less goes into affecting the direction. The limiting condition gives a minimum angle of
\begin{align}
\label{eq:delta_min}
\delta_{\min} &= \acos\left(\frac{\sqrt{s^2 - \flat{v}^2 + \flat{v}_f^2}}{\flat{v}_f} \right).
\end{align}
However, in order to gain the full acceleration, $\flat{a}$, we need the angle to be a bit larger, so we avoid the special case mentioned above. Therefore the optimal angle is the smallest angle at which we receive the full acceleration, $\flat{a}$, which is given by
\begin{align}
\label{eq:delta_opt}
\delta_{\opt} &= \acos\left( \frac{s - \flat{a}}{\flat{v}_f} \right).
\end{align}
As you can see, this optimal angle depends on the current velocity $\flat{v}_f$ and increases towards \ang{90} as the velocity goes to infinity. The other limiting condition (i.e. the maximum angle), shown in Figure \ref{fig:delta_max_air}, is given by
\begin{align}
\label{eq:delta_max}
\delta_{\max} &= \acos\left( \frac{\flat{v}^2 - \flat{v}_f^2 - \flat{a}^2}{2 \flat{a} \flat{v}_f} \right).
\end{align}
Hence, the angles $\delta_{\min}$ and $\delta_{\max}$ represent the boundaries between a speed increase and possible speed decrease. When $\vec{\flat{v}} = \vec{\flat{v}}_f$, e.g. in the air\footnote{In the air there is no friction ($c = 0$).}, equations \eqref{eq:delta_min} and \eqref{eq:delta_max} simplify to respectively
\begin{align}
\label{eq:delta_min_vf}
\delta_{\min} &= \acos\left( \frac{s}{\flat{v}_f} \right),\\
\label{eq:delta_max_vf}
\delta_{\max} &= \acos\left( -\frac{a}{2 \flat{v}} \right).
\end{align}
Note that, in the air, $\delta_{\max}$ exceeds \ang{90} and still results in a speed increase. On the ground, the angles $\delta_{\min}$, $\delta_{\opt}$ and $\delta_{\max}$ will eventually meet when we reach the \emph{maximum ground speed}\footnote{When we introduce velocity snapping (Section \ref{sec:snaphud}), this is no longer the maximum ground speed.}, due to the friction ($c = 6$), shown in Figure \ref{fig:v_ground} (top left magnification). The mathematically derivation is done in Appendix \ref{app:derive_flat_v_max} and results in
\begin{align}
\label{eq:flat_v_max}
\flat{v}_{\max} &= s \sqrt{\frac{A (2 - AT)}{c (2 - cT)}}.
\end{align}
Knowing this, we find that on a flat surface the maximum ground speed for $\text{VQ3} = \qty{409.7180}{ups}$ and $\text{CPM} = \qty{496.5454}{ups}$, at $\qty{125}{fps}$. Note that if there is no friction ($c = 0$), there would in principle be no limit on the maximum speed. This would be an example of being on an icy surface, or in the air. However there are inevitably other aspects of the engine that put limits on very high speeds.

%Figure \ref{fig:delta_phi}, law of cosines
%\begin{align*}
%a^2 &= r^2 + v_f^2 - 2r v_f\cos\varphi,\\
%\cos\varphi &= \frac{r^2 + v_f^2 - a^2}{2r v_f}.
%\end{align*}

\begin{figure}[H]
	\centering
	\begin{subfigure}[t]{.5\textwidth}
		\centering
		\setlength\figureheight{5.5cm}
		\setlength\figurewidth{5.5cm}
		\includetikz{tikz/delta_min_air}
		\caption{}
		\label{fig:delta_min_air}
	\end{subfigure}%
	\begin{subfigure}[t]{.5\textwidth}
		\centering
		\setlength\figureheight{5.5cm}
		\setlength\figurewidth{5.5cm}
		\includetikz{tikz/delta_air}
		\caption{}
		\label{fig:delta_air}
	\end{subfigure}
	\begin{subfigure}[t]{.5\textwidth}
		\centering
		\setlength\figureheight{5.5cm}
		\setlength\figurewidth{5.5cm}
		\includetikz{tikz/delta_opt_air}
		\caption{}
		\label{fig:delta_opt_air}
	\end{subfigure}%
	\begin{subfigure}[t]{.5\textwidth}
		\centering
		\setlength\figureheight{5.5cm}
		\setlength\figurewidth{5.5cm}
		\includetikz{tikz/delta_max_air}
		\caption{}
		\label{fig:delta_max_air}
	\end{subfigure}
	\caption{In the air ($\vec{\flat{v}} = \vec{\flat{v}}_f$). The current velocity $\vec{\flat{v}}$ (\yellowarrow) together with all possible acceleration configurations (\lightorangearea).}
%	\label{fig:delta_air}
\end{figure}

At this point, it might be interesting to define the $\delta$-region where you gain speed in the direction of the current velocity $\uvec{\flat{v}}$. Similar to equation \eqref{eq:delta_min}, we derive the point where we first start gaining speed in the direction $\uvec{\flat{v}}$. The limiting condition gives a minimum angle of
\begin{align}
\label{eq:delta_bar_min}
\bar{\delta}_{\min} &= \acos\left( \frac{s + \sqrt{s^2 - 4 \flat{v}_f (\flat{v} - \flat{v}_f)}}{2 \flat{v}_f} \right).
\end{align}
Similar to equation \eqref{eq:delta_max}, we derive the point where we stop gaining speed in the direction $\uvec{\flat{v}}$. The limiting condition gives a maximum angle of
\begin{align}
\label{eq:delta_bar_max}
\bar{\delta}_{\max} &= \acos\left( \frac{\flat{v} - \flat{v}_f}{\flat{a}} \right).
\end{align}
Hence, the angles $\bar{\delta}_{\min}$ and $\bar{\delta}_{\max}$ represent the boundaries between a speed increase and a speed decrease in the direction of the current velocity $\uvec{\flat{v}}$. Similar as before, when $\vec{\flat{v}} = \vec{\flat{v}}_f$ (e.g. in the air), equations \eqref{eq:delta_bar_min} and \eqref{eq:delta_bar_max} simplify to
\begin{align}
\label{eq:delta_bar_min_vf}
\bar{\delta}_{\min} &= \acos\left( \frac{s}{\flat{v}_f} \right) = \eqref{eq:delta_min_vf},\\
\label{eq:delta_bar_max_vf}
\bar{\delta}_{\max} &= \frac{\pi}{2}.
\end{align}
Knowing all these angles, we can create a CGazHUD (i.e. a strafe-jump helper), which is divided in 5 regions (see Figure \ref{fig:v_air} and \ref{fig:v_ground}):
\begin{itemize}
	\item[\textcolor{cgazgrey!50}{$\blacksquare$}] $\delta \in [0, \delta_{\min})$\\
	The no-acceleration zone.
	\item[\textcolor{cgazgreen!50}{$\blacksquare$}] $\delta \in [\delta_{\min}, \delta_{\opt})$\\
	The zone where we do not get the full acceleration $\flat{a}$ yet, as described above. Note that it is not desired for a human to stay inside this zone.
	\item[\textcolor{cgazdarkgreen!50}{$\blacksquare$}] $\delta \in [\delta_{\opt}, \bar{\delta}_{\max})$\\
	The main acceleration zone. Stay for the most part inside this zone, preferably close to the edge between the two green zones to keep the \emph{change in direction} minimal.
	\item[\textcolor{cgazyellow!50}{$\blacksquare$}] $\delta \in [\bar{\delta}_{\max}, \delta_{\max}]$\\
	The turn zone where we no longer gain speed in the direction of the current velocity $\uvec{\flat{v}}$, however the overall speed increases.
	\item[$\square$] $\delta \in (\delta_{\max}, \pi]$\\
	The deceleration zone.
\end{itemize}

\begin{figure}[H]
	\centering
	\setlength\figureheight{4.8cm}
	\setlength\figurewidth{13cm}
	\includetikz{tikz/v_air}
%	\vspace*{-2.5mm}
	\caption{Numerical example in the air with an initial velocity $\vec{v} = \inlinemat{400, 0, 0}^T$. Magnification $\times30$. CGazHUD boundaries: $0 ~\highlightcgazgrey{\le}~ \delta_{\min}^{\eqref{eq:delta_min_vf}} ~\highlightcgazgreen{\le}~ \delta_{\opt}^{\eqref{eq:delta_opt}} ~\highlightcgazdarkgreen{\le}~ \bar{\delta}_{\max}^{\eqref{eq:delta_bar_max_vf}} ~\highlightcgazyellow{\le}~ \delta_{\max}^{\eqref{eq:delta_max_vf}} \le \pi$.}
	\label{fig:v_air}
\end{figure}
\begin{figure}[H]
	\centering
	\setlength\figureheight{4.8cm}
	\setlength\figurewidth{13cm}
	\includetikz{tikz/v_ground}
	\caption{Numerical example on the ground with an initial velocity $\vec{v} = \inlinemat{400, 0, 0}^T$. Magnification $\times10$. Upper CGazHUD boundaries: $0 ~\highlightcgazgrey{\le}~ \delta_{\min}^{\eqref{eq:delta_min}} ~\highlightcgazgreen{\le}~ \delta_{\opt}^{\eqref{eq:delta_opt}} ~\highlightcgazdarkgreen{\le}~ \bar{\delta}_{\max}^{\eqref{eq:delta_bar_max}} ~\highlightcgazyellow{\le}~ \delta_{\max}^{\eqref{eq:delta_max}} \le \pi$; lower CGazHUD boundaries: $0 ~\highlightcgazgrey{\le}~ \delta_{\min}^{\eqref{eq:delta_min_vf}} ~\highlightcgazgreen{\le}~ \delta_{\opt}^{^{\eqref{eq:delta_opt}}} ~\highlightcgazdarkgreen{\le}~ \bar{\delta}_{\max}^{\eqref{eq:delta_bar_max_vf}} ~\highlightcgazyellow{\le}~ \delta_{\max}^{\eqref{eq:delta_max_vf}} \le \pi$.}
	\label{fig:v_ground}
\end{figure}


\subsection{Strafing on icy surfaces}
\label{sec:slick}
Acceleration $A = 1$ on a icy surface is identical to the acceleration in the air, however, this is not the entire story. Every frame Gravity $g = 800$ (\texttt{g\_gravity}).
\begin{align*}
v_{f_z} - gT
\end{align*}

Explain the drifting speed and sticky jump overbounces (graph time for ob with respect to your current velocity, slick top view turn (with(out) jump)).


\subsection{Strafing on sloped surfaces}
\begin{align*}
\vec{n} &=
\begin{pmatrix}
n_1\\n_2\\n_3
\end{pmatrix}, & \vec{k} &=
\begin{pmatrix}
0\\0\\1
\end{pmatrix},\\
\vec{\flat{n}} &=
\begin{pmatrix}
0\\0\\1
\end{pmatrix}, &
\vec{\bar{k}} &=
\begin{pmatrix}
-n_1\\-n_2\\\hphantom{-}n_3
\end{pmatrix},
\end{align*}
%
\begin{align*}
\mat{P}_{\perp \vec{n} \rightarrow \vec{n}} = \mat{I} - \vec{n}\vec{n}^T &=
\begin{pmatrix}
n_2^2+n_3^2 & -n_1 n_2 & -n_1 n_3\\
-n_1 n_2 & n_1^2+n_3^2 & -n_2 n_3\\
-n_1 n_3 & -n_2 n_3 & n_1^2+n_2^2
\end{pmatrix},\\
\mat{P}_{\perp \vec{n} \rightarrow \vec{k}} = \mat{I} - \frac{\vec{n}\vec{k}^T}{\vec{n}^T\vec{k}} &=
\begin{pmatrix}
1 & 0 & -\frac{n_1}{n_3}\\
0 & 1 & -\frac{n_2}{n_3}\\
0 & 0 & \hphantom{-}0
\end{pmatrix},\\
\mat{P}_{\perp \vec{k} \rightarrow \vec{\bar{k}}} = \mat{I} - \frac{\vec{k}\vec{\bar{k}}^T}{\vec{k}^T\vec{\bar{k}}} &=
\begin{pmatrix}
1 & 0 & 0\\
0 & 1 & 0\\
\frac{n_1}{n_3} & \frac{n_2}{n_3} & 0
\end{pmatrix},
\end{align*}

\begin{align*}
% no slope
\yaw_1 &= -\arctan\frac{n_1}{n_2},\\
% min angle yaw1 - yaw2
\yaw_2 &= -\arctan\frac{n_1}{n_2} + \frac{\pi}{2},\\
% no slope
\yaw_3 &= -\arctan\frac{n_1}{n_2} + \pi,\\
% max angle yaw2 - yaw1
\yaw_4 &= -\arctan\frac{n_1}{n_2} - \frac{\pi}{2},
\end{align*}

It is this acceleration, provided by the movement keys, which gives the player a change in velocity, which ultimately changes his position.
\begin{align*}
\flat{\fmove} &=
\begin{pmatrix}
\cos(\delta+\yaw_4) \\ \sin(\delta+\yaw_4) \\ 0
\end{pmatrix} = \frac{1}{\sqrt{n_1^2 + n_2^2}}
\begin{pmatrix}
\hphantom{-}n_2\sin\delta - n_1\cos\delta\\
-n_1\sin\delta - n_2\cos\delta\\
0
\end{pmatrix},
\end{align*}

One can use the \emph{Rodrigues' rotation formula} to construct the rotation matrix $\mat{R}$ that rotates by an angle $\phi = \arccos(\vec{k}^T\vec{n})$ about the unit vector (\purplearrow)
\begin{equation}
\label{eq:u}
\vec{u} =
\begin{pmatrix}
u_1\\u_2\\u_3
\end{pmatrix} = \frac{1}{\sqrt{n_1^2 + n_2^2}}
\begin{pmatrix}
-n_2\\\hphantom{-}n_1\\\hphantom{-}0
\end{pmatrix}.
\end{equation}
Letting
\[
\mat{W} =
\begin{pmatrix}
\hphantom{-}0 & -u_3 & \hphantom{-}u_2\\
\hphantom{-}u_3 & \hphantom{-}0 & -u_1\\
-u_2 & \hphantom{-}u_1 & \hphantom{-}0
\end{pmatrix},
\]
the Rodrigues' rotation matrix is constructed as
\begin{align*}
\mat{R} &= \mat{I} + \sin \phi \mat{W} + 2\sin^2 \frac{\phi}{2}\mat{W}^2,\\
\mat{R} &= \mat{I} + \sqrt{n_1^2 + n_2^2}\mat{W} + (1 - n_3)\mat{W}^2,\\
\mat{R} &=
\begin{pmatrix}
\frac{n_3 n_1^2 + n_2^2}{n_1^2 + n_2^2} & \frac{n_1 n_2 (n_3 - 1)}{n_1^2 + n_2^2} & n_1\\
\frac{n_1 n_2 (n_3 - 1)}{n_1^2 + n_2^2} & \frac{n_1^2 + n_3 n_2^2}{n_1^2 + n_2^2} & n_2\\
-n_1 & -n_2 & n_3
\end{pmatrix}.
\end{align*}

\begin{align*}
S2F &=
\begin{pmatrix}
\frac{n_3 n_1^2 + n_2^2}{n_1^2 + n_2^2} & \frac{n_1 n_2 (n_3 - 1)}{n_1^2 + n_2^2} & -n_1\\
\frac{n_1 n_2 (n_3 - 1)}{n_1^2 + n_2^2} & \frac{n_1^2 + n_3 n_2^2}{n_1^2 + n_2^2} & -n_2\\
n_1 & n_2 & \hphantom{-}n_3
\end{pmatrix}\\
currentspeed &= S2F velocity;
\end{align*}

\begin{figure}[H]
	\centering
	\begin{subfigure}[t]{\textwidth}
		\centering
		\begin{subfigure}[t]{0.5\textwidth}
			\centering
			\setlength\figureheight{6.5cm}
			\setlength\figurewidth{6.5cm}
			\includetikz{tikz/define}
		\end{subfigure}%
		\begin{subfigure}[t]{0.5\textwidth}
			\centering
			\setlength\figureheight{5cm}
			\setlength\figurewidth{5cm}
			\includetikz{tikz/define2}
		\end{subfigure}
		\caption{}
	\end{subfigure}
	\begin{subfigure}[t]{\textwidth}
		\centering
		\begin{subfigure}[t]{0.5\textwidth}
			\centering
			\setlength\figureheight{6.5cm}
			\setlength\figurewidth{6.5cm}
			\includetikz{tikz/defineT}
		\end{subfigure}%
		\begin{subfigure}[t]{0.5\textwidth}
			\centering
			\setlength\figureheight{5cm}
			\setlength\figurewidth{5cm}
			\includetikz{tikz/defineT2}
		\end{subfigure}
		\caption{}
	\end{subfigure}
	\caption{The unit vector $\uvec{u}$ (\purplearrow) from equation \eqref{eq:u}. (a) Original setup: the key plane (\orangearea) and the surface (\greenarea). (b) Change of basis, with the surface normal $\vec{\flat{n}}$ (\greenarrow) straight up.}
\end{figure}


\subsection{CPM air control}
\label{sec:turnCPM}
Forwardkey aircontrol, forwardkey CGazHUD strafing, forwardkey side speed gain

Sidestrafe aircontrol applies regular acceleration with an additional change of direction. The new velocity is pulled towards the yaw angle.\\
TODO


\subsection{Speed gains from air strafing}
\label{sec:accel_air}
Acceleration from strafing can be obtained using equation \eqref{eq:sAT}.
From below code, acceleration is applied per axis in direction of unit \texttt{wishdir} with magnitude of \texttt{accelspeed}, i.e. $\uvec{\flat{s}}\flat{a}$.
\codeFromFile{firstline=256,lastline=258,gobble=1}{code/game/bg_pmove.c}
Thus, acceleration is dependent on angle $\delta$.\\
W.l.o.g., assume current velocity $\flat{v}$ is solely on the $x$-axis. Then the resulting new velocity $\vec{r}$ becomes
\begin{align*}
\vec{r} =
\begin{pmatrix}
\flat{v} + \flat{a}\cos\delta\\\flat{a}\sin\delta\\0
\end{pmatrix}.
\end{align*}
Given $\delta \in [\delta_{\min}, \delta_{\max}]$, function of new velocity\footnote{When strafing at angle $\delta$ to $\vec{\flat{v}}$ for a single frame.} $r\left(\flat{v}\ ;\delta \right)$ is given by
\begin{align*}
r\left(\flat{v}\ ;\delta \right) &= \sqrt{\left(\flat{v} + \flat{a}\cos\delta \right)^2 + \left(\flat{a}\sin\delta \right)^2}\\
&= \sqrt{\flat{v}^2 + 2\flat{v}\flat{a}\cos\delta + \flat{a}^2}
\end{align*}
and likewise gain in speed $G\left(\flat{v}\ ;\delta \right) = \sqrt{\flat{v}^2 + 2\flat{v}\flat{a}\cos\delta + \flat{a}^2} - \flat{v}$.\\

Now provided that the player is strafing at $\delta_{\opt}$,
\begin{align*}
r\left(\flat{v}\ ;\delta_{\opt} \right) &= \sqrt{\flat{v}^2 + 2\flat{v}\flat{a}\cos{\acos\left( \frac{s - \flat{a}}{\flat{v}} \right)} + \flat{a}^2}\\
&= \sqrt{\flat{v}^2 + 2\flat{v}\flat{a}\frac{s - \flat{a}}{\flat{v}} + \flat{a}^2}\ \ldots\ \frac{s - \flat{a}}{\flat{v}}\le 1
\end{align*}
\begin{align}
\label{eq:accel_opt_air}
\therefore\ r\left(\flat{v}\ ;\delta_{\opt} \right) = \sqrt{\flat{v}^2 + 2s\flat{a} - \flat{a}^2},\ \text{where}\ \flat{v}\ge s - \flat{a}.
\end{align}\\

For normal strafing, $s = 320$, $A = 1$. These values are the same for CPM and VQ3. Hence,
\begin{align*}
\flat{a} = 320\cdot 1\cdot 0.008 = 2.56.
\end{align*}
Using \eqref{eq:accel_opt_air},
\begin{align*}
r_{\text{air-strafe}}\left(\flat{v}\ ;\delta_{\opt} \right) &= \sqrt{\flat{v}^2 + 2\cdot 320 \cdot 2.56 - 2.56^2}\\
&= \sqrt{\flat{v}^2 + 1631.8464},\ \text{where}\ \flat{v}\ge 317.44.
\end{align*}

For CPM sidestrafing, $s = 30$, $A = 70$. Hence, $\flat{a} = 30\cdot 70\cdot 0.008 = 16.8$, and
\begin{align*}
r_{\text{air-sidestrafe}}\left(\flat{v}\ ;\delta_{\opt} \right) &= \sqrt{\flat{v}^2 + 2\cdot 30 \cdot 16.8 - 16.8^2}\\
&= \sqrt{\flat{v}^2 + 725.76},\ \text{where}\ \flat{v}\ge 303.2.
\end{align*}

This also presents the question of whether CPM sidestrafing ever becomes superior to normal strafing.
However, it is clear that $r_{\text{air-strafe}}\left(\flat{v}\ ;\delta_{\opt} \right) > r_{\text{air-sidestrafe}}\left(\flat{v}\ ;\delta_{\opt} \right)\ \forall\ \flat{v} \in \symbb{R}$.


\subsection{Speed gains from ground strafing}
\label{sec:accel_ground}
Deriving formulas for ground strafing works similarly to air strafing, however friction needs to be accounted for.
As friction is applied before acceleration, $\flat{v}_f = (1 - \iota cT)\flat{v}$. Thus, assuming w.l.o.g. $\flat{v}$ is solely on the $x$-axis,
\begin{align*}
\vec{r} =
\begin{pmatrix}
(1 - \iota cT)\flat{v} + \flat{a}\cos\delta\\\flat{a}\sin\delta\\0
\end{pmatrix}.
\end{align*}
Then with $\delta \in [\delta_{\min}, \delta_{\max}]$,
\begin{align*}
r\left(\flat{v}\ ;\delta \right) = \sqrt{\left((1 - \iota cT)\flat{v} \right)^2 + 2(1 - \iota cT)\flat{v}\flat{a}\cos\delta + \flat{a}^2}
\end{align*}
and likewise gain in speed $G\left(\flat{v}\ ;\delta \right) = \sqrt{\left((1 - \iota cT)\flat{v} \right)^2 + 2(1 - \iota cT)\flat{v}\flat{a}\cos\delta + \flat{a}^2} - \flat{v}$.\\

Now provided that the player is strafing at $\delta_{\opt}$,
\begin{align}
\label{eq:accel_opt_ground}
r\left(\flat{v}\ ;\delta_{\opt} \right) = \sqrt{\left((1 - \iota cT)\flat{v} \right)^2 + 2s\flat{a} - \flat{a}^2},\ \text{where}\ \flat{v}_f\ge s - \flat{a}.
\end{align}\\

For CPM, $\flat{a} = 320\cdot 15\cdot 0.008 = 38.4$. Hence by \eqref{eq:accel_opt_ground},
\begin{align*}
r_{\text{CPM-ground-strafe}}\left(\flat{v}\ ;\delta_{\opt} \right) &= \sqrt{\left((1 - \iota cT)\flat{v} \right)^2 + 23101.44},\ \text{where}\ \flat{v}_f\ge 281.6.
\end{align*}
For VQ3, $\flat{a} = 320\cdot 10\cdot 0.008 = 25.6$. Hence,
\begin{align*}
r_{\text{VQ3-ground-strafe}}\left(\flat{v}\ ;\delta_{\opt} \right) &= \sqrt{\left((1 - \iota cT)\flat{v} \right)^2 + 15728.64},\ \text{where}\ \flat{v}_f\ge 294.4.
\end{align*}\\

\eqref{eq:accel_opt_ground} can also be used to derive $\flat{v}_{\max}$ from \eqref{eq:flat_v_max}, done in Appendix \ref{app:derive_flat_v_max_alternative}.


\subsection{Swimming and flying}
\label{sec:swim_and_flying}
After friction is applied, the current velocity vector becomes
\begin{align*}
\vec{v}_f &=
\begin{pmatrix}
v_{fx} \\ v_{fy} \\ v_{fz}
\end{pmatrix} = \norm*{\vec{v}_f} \uvec{v} = (1-cT)\vec{v},
\end{align*}
with a magnitude
\begin{align*}
%\label{eq:vf}
\norm*{\vec{v}_f} &= \sqrt{\vec{v}_f^T \vec{v}_f} = \sqrt{v_{fx}^2 + v_{fy}^2 + v_{fz}^2} = v_f = (1-cT)v,
\end{align*}
with $c = 3$ the friction factor when using flight (\texttt{pm\_flightfriction}).

\begin{align*}
\vec{\fmove} &= \norm{\vec{\fmove}} \uvec{\fmove} = \fmove
\begin{pmatrix}
\cos\rho\cos\gamma \\ \cos\rho\sin\gamma \\ -\sin\rho
\end{pmatrix},\\
%
\vec{\rmove} &= \norm{\vec{\rmove}} \uvec{\rmove} = \rmove
\begin{pmatrix}
\hphantom{-}\sin\gamma \\ -\cos\gamma \\ 0
\end{pmatrix},\\
%
\vec{\umove} &= \norm{\vec{\umove}} \uvec{\umove} = \umove
\begin{pmatrix}
\sin\rho\cos\gamma \\ \sin\rho\sin\gamma \\ \cos\rho
\end{pmatrix},
\end{align*}

\begin{align*}
\texttt{wishvel} &= \vec{\fmove} + \vec{\rmove} + \umove
\begin{pmatrix}
0\\0\\1
\end{pmatrix} =
\begin{pmatrix}
\fmove \cos\rho\cos\gamma + \rmove \sin\gamma\\
\fmove \cos\rho\sin\gamma - \rmove \cos\gamma\\
-\fmove \sin\rho + \umove
\end{pmatrix},\\
\vec{\fmove} + \vec{\rmove} + \vec{\umove} &=
\begin{pmatrix}
(\fmove \cos\rho + \umove \sin\rho)\cos\gamma + \rmove \sin\gamma\\
(\fmove \cos\rho + \umove \sin\rho)\sin\gamma - \rmove \cos\gamma\\
-\fmove \sin\rho + \umove \cos\rho
\end{pmatrix},
% = \frac{1}{\flat{v}}
%\begin{pmatrix}
%(\hphantom{-}\fmove v_x + \rmove v_y)\cos\delta + (\rmove v_x - \fmove v_y)\sin\delta\\
%(-\rmove v_x + \fmove v_y)\cos\delta + (\fmove v_x + \rmove v_y)\sin\delta\\
%0
%\end{pmatrix}
\end{align*}
\begin{align*}
\norm{\texttt{wishvel}} &= \sqrt{(\fmove \cos\rho)^2 + \rmove^2 + (-\fmove \sin\rho + \umove)^2} = \sqrt{\fmove^2 + \rmove^2 + \umove^2  - 2\fmove\umove\sin\rho},\\
\norm{\vec{\fmove} + \vec{\rmove} + \vec{\umove}} &= \sqrt{(\fmove \cos\rho + \umove \sin\rho)^2 + \rmove^2 + (-\fmove \sin\rho + \umove \cos\rho)^2} = \sqrt{\fmove^2 + \rmove^2 + \umove^2} = \norm{\vec{\cmd}},
\end{align*}
Using the law of cosines $\fumove^2 = \fmove^2 + \umove^2 - 2\fmove\umove\cos\left(\frac{\pi}{2} - \rho \right)$ and
\begin{align*}
\norm{\texttt{wishvel}} &= \sqrt{\fumove^2 + \rmove^2},
\end{align*}

\begin{align*}
\texttt{wishdir} &= \uvec{a} = \frac{\texttt{wishvel}}{\norm{\texttt{wishvel}}} = \frac{1}{\sqrt{\fmove^2 + \rmove^2 + \umove^2  - 2\fmove\umove\sin\rho}}
\begin{pmatrix}
\fmove \cos\rho\cos\gamma + \rmove \sin\gamma\\
\fmove \cos\rho\sin\gamma - \rmove \cos\gamma\\
-\fmove \sin\rho + \umove
\end{pmatrix},
%\frac{\vec{\fmove} + \vec{\rmove} + \vec{\umove}}{\norm{\vec{\fmove} + \vec{\rmove} + \vec{\umove}}} &=
%%\frac{1}{\flat{\cmd} \flat{v}}
%%\begin{pmatrix}
%%(\hphantom{-}\fmove v_x + \rmove v_y)\cos\delta + (\rmove v_x - \fmove v_y)\sin\delta\\
%%(-\rmove v_x + \fmove v_y)\cos\delta + (\fmove v_x + \rmove v_y)\sin\delta\\
%%0
%%\end{pmatrix}
%\frac{1}{\cmd}
%\begin{pmatrix}
%(\fmove \cos\rho + \umove \sin\rho)\cos\gamma + \rmove \sin\gamma\\
%(\fmove \cos\rho + \umove \sin\rho)\sin\gamma - \rmove \cos\gamma\\
%-\fmove \sin\rho + \umove \cos\rho
%\end{pmatrix},
\end{align*}

\begin{align*}
\texttt{wishspeed} &= s = 320\frac{\norm{\vec{\cmd}}_{\infty}}{127\hphantom{_{\infty}}} \frac{\norm{\texttt{wishvel}}}{\norm{\vec{\cmd}}} = 320\frac{\norm{\vec{\cmd}}_{\infty}}{127\hphantom{_{\infty}}} \frac{\sqrt{\fmove^2 + \rmove^2 + \umove^2  - 2\fmove\umove\sin\rho}}{\norm{\vec{\cmd}}},\\
\norm{\vec{a}} &= a = sAT = 320AT\frac{\norm{\vec{\cmd}}_{\infty}}{127\hphantom{_{\infty}}} \frac{\norm{\texttt{wishvel}}}{\norm{\vec{\cmd}}},
\end{align*}
with $A = 8$ (\texttt{pm\_flyaccelerate}).

However, in order to gain the full acceleration, $a$, we need the angle to be a bit larger, so we avoid the special case mentioned above. Therefore the optimal angle is the smallest angle at which we receive the full acceleration, $a$, which is given by
\begin{align*}
s(1-AT) = \vec{v}_f^T \uvec{a},\\
320\frac{\norm{\vec{\cmd}}_{\infty}}{127\hphantom{_{\infty}}} \frac{\cmd^2  - 2\fmove\umove\sin\rho}{\cmd}(1-AT) &= v_{fx}(\fmove \cos\rho\cos\gamma + \rmove \sin\gamma) + v_{fz}(-\fmove \sin\rho + \umove)
\end{align*}

\begin{align}
\label{eq:ABCD}
A + B\sin\rho + C\sin\gamma + D\cos\rho\cos\gamma = 0,
\end{align}

\begin{align*}
S &= \frac{320\norm{\vec{\cmd}}_{\infty}(1-AT)}{127},\\
A &= S\cmd         - v_{fz}\umove,\\
B &= -\frac{2S}{\cmd}\fmove\umove + v_{fz}\fmove,\\
C &= -v_{fx}\rmove,\\
D &= -v_{fx}\fmove,
\end{align*}

\begin{verbatim}
tmp = D^2*cos(y)^2 - (A + C*sin(y) - B)*(A + C*sin(y) + B);
tmp(tmp < 0) = nan;
tmp1 = -2*atan2((B - sqrt(tmp)),(A + C*sin(y) - D*cos(y)));
tmp2 = -2*atan2((B + sqrt(tmp)),(A + C*sin(y) - D*cos(y)));
tmp1(tmp1 >  pi) = tmp1(tmp1 >  pi) - 2*pi;
tmp1(tmp1 < -pi) = tmp1(tmp1 < -pi) + 2*pi;
tmp2(tmp2 >  pi) = tmp2(tmp2 >  pi) - 2*pi;
tmp2(tmp2 < -pi) = tmp2(tmp2 < -pi) + 2*pi;
\end{verbatim}

\begin{align*}
\norm*{\vec{v}_f + \norm{\vec{a}}\uvec{a}}^2 &= \norm{\vec{v}}^2,\\
2\vec{a}^T \vec{v}_f + \vec{a}^T \vec{a} &= \vec{v}^T \vec{v} - \vec{v}_f^T \vec{v}_f,\\
2\norm{\vec{a}} \vec{v}_f^T \uvec{a} + \norm{\vec{a}}^2 &= \norm{\vec{v}}^2 - \norm*{\vec{v}_f}^2,\\
320AT\frac{\norm{\vec{\cmd}}_{\infty}}{127\hphantom{_{\infty}}} \frac{1}{\cmd}2 (v_{fx}(\fmove \cos\rho\cos\gamma + \rmove \sin\gamma) + v_{fz}(-\fmove \sin\rho + \umove) ) +\\
320^2 A^2 T^2\frac{\norm{\vec{\cmd}}^2_{\infty}}{127^2\hphantom{_{\infty}}} \frac{\cmd^2  - 2\fmove\umove\sin\rho}{\cmd^2} &= v_x^2 + v_z^2 - v_{fx}^2 - v_{fz}^2,
\end{align*}

\begin{align*}
S' &= \frac{320AT\norm{\vec{\cmd}}_{\infty}}{127} = \frac{320\norm{\vec{\cmd}}_{\infty}}{127} - S,\\
S &= \frac{320\norm{\vec{\cmd}}_{\infty}}{127} - S',\\
A' &= \hphantom{-}\frac{2S'}{\cmd}v_{fz}\umove + S'^2 - v_x^2 - v_z^2 + v_{fx}^2 + v_{fz}^2 = A - S\cmd + \left(\frac{2S'}{\cmd} + 1\right)v_{fz}\umove + S'^2 - v_x^2 - v_z^2 + v_{fx}^2 + v_{fz}^2,\\
B' &= -\frac{2S'}{\cmd}v_{fz}\fmove - \frac{2S'^2}{\cmd^2}\fmove\umove,\\
C' &= \hphantom{-}\frac{2S'}{\cmd}v_{fx}\rmove = -\frac{2S'}{\cmd}C,\\
D' &= \hphantom{-}\frac{2S'}{\cmd}v_{fx}\fmove = -\frac{2S'}{\cmd}D,
\end{align*}
\begin{align*}
S' &= \frac{320AT\norm{\vec{\cmd}}_{\infty}}{127},\\
S &= \frac{320\norm{\vec{\cmd}}_{\infty}}{127} - S',\\
-\frac{2S'}{\cmd}\\
A' &= -v_{fz}\umove - \frac{\cmd}{2S'}(S'^2 - v_x^2 - v_z^2 + v_{fx}^2 + v_{fz}^2),\\
B' &= -\frac{S'}{\cmd}\fmove\umove + v_{fz}\fmove,\\
C' &= -v_{fx}\rmove,\\
D' &= -v_{fx}\fmove,
\end{align*}

\begin{align*}
\frac{d}{d\rho}\norm*{\vec{v}_f + \norm{\vec{a}}\uvec{a}}^2 &= 0,\\
\frac{d}{d\rho}2\vec{a}^T \vec{v}_f + \frac{d}{d\rho}\vec{a}^T \vec{a} &= 0,\\
\frac{d}{d\rho}2\norm{\vec{a}} \vec{v}_f^T \uvec{a} + \frac{d}{d\rho}\norm{\vec{a}}^2 &= 0,\\
\frac{d}{d\rho}320AT\frac{\norm{\vec{\cmd}}_{\infty}}{127\hphantom{_{\infty}}} \frac{1}{\cmd}2 (v_{fx}(\fmove \cos\rho\cos\gamma + \rmove \sin\gamma) + v_{fz}(-\fmove \sin\rho + \umove) ) +\\
\frac{d}{d\rho}320^2 A^2 T^2\frac{\norm{\vec{\cmd}}^2_{\infty}}{127^2\hphantom{_{\infty}}} \frac{\cmd^2  - 2\fmove\umove\sin\rho}{\cmd^2} &= 0,\\
%
\frac{d}{d\rho}320AT\frac{\norm{\vec{\cmd}}_{\infty}}{127\hphantom{_{\infty}}} \frac{1}{\cmd}2 (v_{fx}\fmove \cos\rho\cos\gamma - v_{fz}\fmove \sin\rho ) - \frac{d}{d\rho}320^2 A^2 T^2\frac{\norm{\vec{\cmd}}^2_{\infty}}{127^2\hphantom{_{\infty}}} \frac{2\fmove\umove}{\cmd^2}\sin\rho &= 0,\\
-320AT\frac{\norm{\vec{\cmd}}_{\infty}}{127\hphantom{_{\infty}}} \frac{1}{\cmd}2 (v_{fx}\fmove \sin\rho\cos\gamma + v_{fz}\fmove \cos\rho ) - 320^2 A^2 T^2\frac{\norm{\vec{\cmd}}^2_{\infty}}{127^2\hphantom{_{\infty}}} \frac{2\fmove\umove}{\cmd^2}\cos\rho &= 0,
\end{align*}

\begin{align*}
A\sin\rho\cos\gamma + B\cos\rho &= 0,
\end{align*}

\begin{align*}
S &= \frac{320AT\norm{\vec{\cmd}}_{\infty}}{127},\\
A &= -\frac{2S}{\cmd}v_{fx}\fmove,\\
B &= -\frac{2S}{\cmd}v_{fz}\fmove - \frac{2S^2}{\cmd^2}\fmove\umove,
\end{align*}

\begin{align*}
\frac{d}{d\rho}\norm*{\vec{v}_f + \norm{\vec{a}}\uvec{a}}^2 &= 0,\\
\frac{d}{d\rho}2\vec{a}^T \vec{v}_f + \frac{d}{d\rho}\vec{a}^T \vec{a} &= 0,\\
\frac{d}{d\rho}2\norm{\vec{a}} \vec{v}_f^T \uvec{a} + \frac{d}{d\rho}\norm{\vec{a}}^2 &= 0,\\
\frac{d}{d\gamma}320AT\frac{\norm{\vec{\cmd}}_{\infty}}{127\hphantom{_{\infty}}} \frac{1}{\cmd}2 (v_{fx}(\fmove \cos\rho\cos\gamma + \rmove \sin\gamma) + v_{fz}(-\fmove \sin\rho + \umove) ) +\\
\frac{d}{d\gamma}320^2 A^2 T^2\frac{\norm{\vec{\cmd}}^2_{\infty}}{127^2\hphantom{_{\infty}}} \frac{\cmd^2  - 2\fmove\umove\sin\rho}{\cmd^2} &= 0,\\
%
\frac{d}{d\gamma}320AT\frac{\norm{\vec{\cmd}}_{\infty}}{127\hphantom{_{\infty}}} \frac{1}{\cmd}2 (v_{fx}(\fmove \cos\rho\cos\gamma + \rmove \sin\gamma)) &= 0,\\
-320AT\frac{\norm{\vec{\cmd}}_{\infty}}{127\hphantom{_{\infty}}} \frac{1}{\cmd}2 (v_{fx}(\fmove \cos\rho\sin\gamma - \rmove \cos\gamma)) &= 0,
\end{align*}

\begin{align*}
A\cos\rho\sin\gamma + B\cos\gamma &= 0,
\end{align*}

\begin{align*}
S &= \frac{320AT\norm{\vec{\cmd}}_{\infty}}{127},\\
A &= -\frac{2S}{\cmd}v_{fx}\fmove,\\
B &= \hphantom{-}\frac{2S}{\cmd}v_{fx}\rmove,
\end{align*}

\begin{align*}
B\cos\rho - D\sin\rho\cos\gamma &= 0,\\
C\cos\gamma - D\cos\rho\sin\gamma &= 0,
\end{align*}

Grey area:
\begin{align*}
s = \vec{v}_f^T \uvec{a},\\
320\frac{\norm{\vec{\cmd}}_{\infty}}{127\hphantom{_{\infty}}} \frac{\cmd^2  - 2\fmove\umove\sin\rho}{\cmd} &= v_{fx}(\fmove \cos\rho\cos\gamma + \rmove \sin\gamma) + v_{fz}(-\fmove \sin\rho + \umove)
\end{align*}
\eqref{eq:ABCD}
\begin{align*}
S &= \frac{320\norm{\vec{\cmd}}_{\infty}}{127},\\
A &= S\cmd         - v_{fz}\umove,\\
B &= -\frac{2S}{\cmd}\fmove\umove + v_{fz}\fmove,\\
C &= -v_{fx}\rmove,\\
D &= -v_{fx}\fmove,
\end{align*}

\texttt{PM\_Accelerate (wishdir, wishspeed, pm\_flyaccelerate);}

\begin{align*}
\norm*{\vec{v}_f + (s - \vec{v}_f^T \uvec{a})\uvec{a}}^2 &= \norm{\vec{v}}^2,\\
(\vec{v}_f + (s - \vec{v}_f^T \uvec{a})\uvec{a})^T (\vec{v}_f + (s - \vec{v}_f^T \uvec{a})\uvec{a}) &= \vec{v}^T \vec{v},\\
2(s - \vec{v}_f^T \uvec{a})\uvec{a}^T \vec{v}_f + (s - \vec{v}_f^T \uvec{a})^2\uvec{a}^T\uvec{a} &= \vec{v}^T \vec{v} - \vec{v}_f^T \vec{v}_f,\\
2(s - \vec{v}_f^T \uvec{a})\vec{v}_f^T \uvec{a} + (s - \vec{v}_f^T \uvec{a})^2 &= \norm{\vec{v}}^2 - \norm*{\vec{v}_f}^2,\\
s^2 - \left(\vec{v}_f^T \uvec{a}\right)^2 &= v^2 - v_f^2,\\
320^2\frac{\norm{\vec{\cmd}}_{\infty}^2}{127^2\hphantom{_{\infty}}} \frac{(\cmd^2  - 2\fmove\umove\sin\rho)^2}{\cmd^2} -\\ (v_{fx}(\fmove \cos\rho\cos\gamma + \rmove \sin\gamma) + v_{fz}(-\fmove \sin\rho + \umove))^2 &= (v_x^2 + v_z^2 - v_{fx}^2 - v_{fz}^2) (\cmd^2  - 2\fmove\umove\sin\rho),
\end{align*}
which is too complex to calculate efficiently.

\section{SnapHUD}
\label{sec:snaphud}

\subsection{Basics}
\label{sec:snap_basics}
In order to save resources, the game engine rounds each component of some vectors each frame, as shown by the below code. Albeit a simple change, this largely changes the way the game is played.
\codeFromFile{firstline=151,lastline=156}{code/unix/unix_shared.c}
Since player velocity is one of the vectors that is rounded, we can interpret this as rounding the speed change the player receives each frame, forming
\begin{align*}
\round{\vec{a}} &=
\begin{pmatrix}
\round{a_x} \\ \round{a_y} \\ \round{a_z}
\end{pmatrix}, & \round{\vec{r}} &=
\begin{pmatrix}
\round{r_x} \\ \round{r_y} \\ \round{r_z}
\end{pmatrix}.
\end{align*}
Hence, there are specific constant view yaw angle ranges called ``snap zones'' which provide the same acceleration regardless of where the player is looking in the zone.
Some of these zones are rounded to give a value $\round*{\flat{a}} > sAT$, and similarly some are rounded such that $\round*{\flat{a}} < sAT$.
Note that the player will still only accelerate if $\delta \in [\delta_{\min}, \delta_{\max}]$.\\
Additionally, the closer $\delta_{\min}$ is to the border of a snap zone, there will be a greater change in speed as the acceleration is pointed more in the direction of $\vec{\flat{v}}$, meaning more of the acceleration goes into increasing the speed rather than changing the direction.
This also means that as $\delta_{\min}$ gets closer to the border of a snap zone, $\delta_{\min}$ will change faster due to speed increasing more rapidly.
Players often attempt to optimise (by decreasing) the distance between $\delta_{\min}$ and the border of the snap zone in order to maximise acceleration through techniques referred to as ``snap manipulation'' or ``snap zone manipulation.''\\

Furthermore, due to velocity snapping rotating the acceleration vector, given a sufficiently high velocity it is possible to lose speed while strafing with $\delta \in [\delta_{\min}, \delta_{\max}]$, which is discussed in Section \ref{sec:speed_loss_while_strafing}.\\

We can form an illustration of the snap zones in any conditions given that we know the acceleration the player is receiving in that frame.
In normal conditions while air strafing at $\qty{125}{fps}$ with no moveup and $\delta \in [\delta_{\opt}, \delta_{\max}]$, $\flat{a} = 2.56$. The effect of the rounding can be visualised with a circle with radius 2.56.\\

TODO


\subsection{Frame rate impact}
\label{sec:snap_frame_rate}
TODO


\subsection{Proper maximum ground speed}
\label{sec:max_ground_speed}
The addition of velocity snapping also affects the maximum ground speed previously calculated to be $\flat{v}_{\max}$.
There are technically two solutions to the proper maximum ground speed for CPM and VQ3: allowing the player to turn any direction (clockwise and anticlockwise), or restricting the player's rotation to one direction as is done in a circle jump.\\
TODO: method\\

Results\\
\begin{tabular*}{\textwidth}{c @{\extracolsep{\fill}} cc}
Restricted && Unrestricted\\
& CPM results &\\
$\flat{v}_{\max\_\proper} = \norm*{\begin{pmatrix}
TODO \\ TODO \\ 0
\end{pmatrix}} = \qty{498.787}{ups}$ && $\flat{v}_{\max\_\proper} = \norm*{\begin{pmatrix}
TODO \\ TODO \\ 0
\end{pmatrix}} = \qty{500.3}{ups}$\\
& VQ3 results &\\
$\flat{v}_{\max\_\proper} = \norm*{\begin{pmatrix}
TODO \\ TODO \\ 0
\end{pmatrix}} = TODO$ && $\flat{v}_{\max\_\proper} = \norm*{\begin{pmatrix}
403 \\ 91 \\ 0
\end{pmatrix}} = \qty{413.146}{ups}$
\end{tabular*}


\subsection{Speed loss while strafing optimally}
\label{sec:speed_loss_while_strafing}
As mentioned before, given a high enough velocity the player can lose speed while strafing with $\delta \in [\delta_{\min}, \delta_{\max}]$, due to the acceleration being rounded to an angle greater than $\delta_{\max}$. CPM players often attempt to counteract this through snap zone manipulation, by W-turning through such areas of speed loss.\\

While air strafing, the lowest velocity that such speed loss can occur can be calculated. First it would be useful to define \emph{symmetrical snap zones} as snap zones where the respective $\round*{\vec{\flat{a}}}$ vector intersects the mean of the ranges of angles of the snap zone. It does not matter which direction these zones are approached from. There are 4 symmetrical snap zones when $0.5\le \flat{a} < \sqrt{0.5}$ and 8 when $\flat{a}\ge \sqrt{0.5}$, logically being the snap zones along the axes and diagonals.
Similarly, \emph{asymmetrical snap zones} are snap zones where this is not the case.
A loss in speed from strafing can first occur in a zone with the largest difference (in terms of angle) between a border of such snap zone and the respective $\round*{\vec{\flat{a}}}$ for that zone.\\

In normal conditions of $\qty{125}{fps}$ while air strafing with no upmove, this difference $\Delta$ is equal to
\begin{align*}
\atan\left(\frac{1}{2} \right) - \atan\left(\frac{\sqrt{\flat{a}^2 - 2.5^2}}{2.5} \right) = \qty{0.246716}{rad},
\end{align*}
and occurs when entering the big uncoloured snap zone from the small zone,\footnote{TODO: explain each of the 24 snap zones} and hence $\round*{\flat{a}} = \sqrt{1^2 + 2^2} = \sqrt{5}$. Since velocity snapping changes the received acceleration and a proper $\delta_{\max}$ value depends on the acceleration after velocity snapping, we must first define $\delta_{\max\_\proper}$ as
\begin{align}
\delta_{\max\_\proper} &= \acos\left(\frac{\flat{v}^2 - \flat{v}_f^2 - \round*{\flat{a}}^2}{2\round*{\flat{a}}\flat{v}_f} \right)\\\nonumber
&= \acos\left(-\frac{\round*{\flat{a}}}{2\flat{v}} \right)\ \ldots\ \text{while in air.}
\end{align}
Note that definitions of all other proper $\delta$ variables are situation-based so a formula similar to this cannot realistically be derived.
Assuming the player is strafing at $\delta_{\opt}$, velocities at which speed can be lost by strafing can hence be calculated by solving
\begin{align*}
\delta_{\opt} + \Delta &> \delta_{\max\_\proper}\\
\acos\left(\frac{s - \flat{a}}{\flat{v}} \right) + \Delta &> \acos\left(-\frac{\round*{\flat{a}}}{2\flat{v}} \right),
\end{align*}
giving $\flat{v} > \qty{1304.247}{ups}$, which first occurs with
\begin{align*}
\vec{v} =
\begin{pmatrix}
1278 \\ 282 \\ v_z
\end{pmatrix}.
\end{align*}
It should be noted this solution is frame rate dependent, and that the solution can be reflected about the $x$ and $y$ axes and diagonals to produce 7 other equivalent solutions.

\section{TODO}
\label{sec:todo}

\begin{enumerate}
  \item $a$ is not really acceleration, more like additional speed (check units)
  \item $AT \le 1$ and $\iota cT \le 1$
  \item OBS, zwob, multiple, sticky
  \item wallbugs
  \item frame rate dependency/impact
  \item playbox hitbox sizes/origin location
  \item skimtime: not enough -z ups
  \item wallstrafe (2 keys better up to a certain point)
  \item plasma climbing (i.e. optimal angle 70-73 (\url{https://github.com/id-Software/Quake-III-Arena/blob/master/code/game/g_weapon.c#L140} SnapVectorTowards))
  \item ceiling bugs (\url{https://github.com/id-Software/Quake-III-Arena/blob/master/code/game/g_combat.c#L1189})
  \item step-up
\end{enumerate}


\begin{appendices}
	\section{Definitions}
\label{app:definitions}

\subsection*{Vector}
\cite{dot_product}\\
An $n$-dimensional vector is represented as
\[
\lcr{}{
\vec{x} =
\begin{pmatrix}
x_1 \\ x_2 \\ \vdots \\ x_n
\end{pmatrix} = \norm{\vec{x}} \uvec{x},
}{
\forall i \in \set{1, 2, ..., n}, x_i \in \mathbb{R},
}
\]
where $x_1, x_2, ..., x_n$ are the components of $\vec{x}$. Alternatively, the vector is uniquely defined by its magnitude and direction. The magnitude is described by the \emph{norm}%
\footnote{The norm $\norm{\cdot}$ represents the Euclidean norm, i.e. $\ell^2$-norm, unless specified otherwise.}
\[
\norm{\vec{x}} = \sqrt{\vec{x}^T \vec{x}} = \sqrt{\sum_{i=1}^{n} x_i^2} = x,
\]
and its direction with the \emph{unit vector}
\[
\lcr{}{
\uvec{x} = \frac{\vec{x}}{\norm{\vec{x}}},
}{
\norm{\uvec{x}} = 1.
}
\]
The \emph{dot product} of two vectors $\vec{a} = \inlinemat{a_1, a_2, \cdots, a_n}^T$ and $\vec{b} = \inlinemat{b_1, b_2, \cdots, b_n}^T$ can be defined both algebraically
\[
\vec{a} \cdot \vec{b} = \vec{a}^T \vec{b} = \sum_{i=1}^{n} a_i b_i,
\]
and geometrically
\[
\vec{a} \cdot \vec{b} = \norm{\vec{a}} \norm{\vec{b}} \cos{\theta},
\]
with $\theta$ the angle between the vectors. Note that when $\vec{a}$ and $\vec{b}$ are orthogonal, i.e. $\theta = \pi/2$, the dot product simplifies to
\[
\vec{a} \cdot \vec{b} = 0.
\] 
%On the other hand, when the vectors are codirectional (i.e. $\theta = 0$), the dot product becomes
%\[
%\vec{a} \cdot \vec{b} = \norm{\vec{a}} \norm{\vec{b}}.
%\]

\subsection*{Normal}
\cite{euclidean_space,hyperplane,normal}\\
A hyperplane, which is a flat $(n-1)$-dimensional subset of the $n$-dimensional Euclidean space, can be described by $n-1$ linearly independent in-plane vectors $\vec{x}_1, \vec{x}_2, ..., \vec{x}_{n-1}$ together with a point on this hyperplane $\vec{x}_0$. Hence, its parameterization
\[
\vec{r}(\vec{t}) = \vec{x}_0 + \mat{X} \vec{t},
\]
with $\mat{X} = \inlinemat{\vec{x}_1, \vec{x}_2, \cdots, \vec{x}_{n-1}}$ and $\vec{t} = \inlinemat{t_1, t_2, \cdots, t_{n-1}}^T$. Any vector in the kernel of $\mat{X}$, i.e. $\mat{X} \vec{n} = \vec{0}$, is perpendicular to the hyperplane and is called a \emph{normal}.
%linear equation hyperplane?

\subsection*{Projection}
%\ref{app:projection}
\cite{projection_matrix}\\
An orthogonal projection can be performed by a matrix multiplication. To project \emph{onto} the vector $\vec{x}$, the matrix is defined by the outer product
\begin{align*}
\lcr{}{
\projmat{x} = \uvec{x} \uvec{x}^T = \projmat{x}^T =
\projmat{x}^k,
}{
k \in \mathbb{N}^+.
}
\end{align*}
Similarly, to project \emph{along} the vector $\vec{x}$ onto its corresponding hyperplane, the matrix is defined by
\begin{align*}
\lcr{}{
\resprojmat{x} = \mat{I} - \projmat{x} = \resprojmat{x}^T = \resprojmat{x}^k,
}{
k \in \mathbb{N}^+,
}
\end{align*}
with $\mat{I}$ the identity matrix.
%sometimes referred to as \emph{residual maker matrix}

	\section{Code snippets}
\label{app:code}

\subsection{\texttt{AngleVectors}}
\label{app:angle_vectors}
\codeFromFile{firstline=1238,lastline=1271}{code/game/q_math.c}


\subsection{\texttt{PM\_Friction}}
\label{app:friction}
\codeFromFile{firstline=165,lastline=230}{code/game/bg_pmove.c}


\subsection{\texttt{PM\_Accelerate}}
\label{app:accelerate}
\codeFromFile{firstline=223,lastline=277}{code/game/bg_pmove.c}


\subsection{\texttt{PM\_CmdScale}}
\label{app:cmd_scale}
\codeFromFile{firstline=281,lastline=311}{code/game/bg_pmove.c}


\subsection{\texttt{PM\_ClipVelocity}}
\label{app:clip_velocity}
\codeFromFile{firstline=138,lastline=162}{code/game/bg_pmove.c}

	\section{Derivations}
\label{app:derivations}
Common assumptions in these derivations
\begin{align}
\label{eq:assump_accel_1}
0 \le \mathmakebox[\widthof{$\iota c$}]{A}T \le 1,\\
\label{eq:assump_friction_1}
0 \le \iota cT \le 1,
\end{align}
hence the following also holds
\begin{align}
\label{eq:assump_accel_2}
0 \le \mathmakebox[\widthof{$\iota c$}]{A}T (2 - \mathmakebox[\widthof{$\iota c$}]{A} T) \le 1,\\
\label{eq:assump_friction_2}
0 \le \iota cT(2 - \iota cT) \le 1.
\end{align}


\subsection{\texorpdfstring{$\delta_{\min}$}{delta\_min}}
\eqref{eq:delta_min}
\begin{align*}
\norm{\vec{\flat{v}}_f + (s - \flat{v}_f \cos\delta_{\min})\uvec{\flat{a}}}^2 &= \norm{\vec{\flat{v}}}^2,\\
(\vec{\flat{v}}_f + (s - \flat{v}_f \cos\delta_{\min})\uvec{\flat{a}})^T (\vec{\flat{v}}_f + (s - \flat{v}_f \cos\delta_{\min})\uvec{\flat{a}}) &= \vec{\flat{v}}^T \vec{\flat{v}},\\
2(s - \flat{v}_f \cos\delta_{\min}) \uvec{\flat{a}}^T \vec{\flat{v}}_f + (s - \flat{v}_f \cos\delta_{\min})^2 \uvec{\flat{a}}^T \uvec{\flat{a}} &= \vec{\flat{v}}^T \vec{\flat{v}} - \vec{\flat{v}}_f^T \vec{\flat{v}}_f,\\
2(s - \flat{v}_f \cos\delta_{\min}) \flat{v}_f \cos\delta_{\min} + (s - \flat{v}_f \cos\delta_{\min})^2 &= \flat{v}^2 - \flat{v}_f^2,\\
s^2 - \flat{v}_f^2 \cos^2\delta_{\min} &= \flat{v}^2 - \flat{v}_f^2,\\
\delta_{\min} &= \acos\left(\frac{\sqrt{s^2 - \flat{v}^2 + \flat{v}_f^2}}{\flat{v}_f} \right).
\end{align*}
Alternatively, using equation \eqref{eq:flat_vf} we find the extended form
\begin{align*}
\delta_{\min} &= \acos\left(\frac{\sqrt{s^2 - \flat{v}^2 + \flat{v}_f^2}}{\flat{v}_f} \right).\\
&= \acos\left(\frac{\sqrt{s^2 - \flat{v}^2 + (\flat{v} - \flat{v}\iota cT)^2}}{\flat{v} - \flat{v}\iota cT} \right)\\
&= \acos\left(\sqrt{\frac{s^2 - \flat{v}^2\iota cT(2 - \iota cT)}{\flat{v}^2 - \flat{v}^2\iota cT(2 - \iota cT)}} \right)\\
\end{align*}

\begin{itemize}
\item
Take square root of positive number + \eqref{eq:assump_friction_2}
\begin{align}
\nonumber
\flat{v}^2 \iota cT (2 - \iota cT) &\le s^2,\\
\nonumber
\begin{cases}
\flat{v}^2 \le s^2, &\flat{v} \le d cT\\
d cT (2 \flat{v} - d cT) \le s^2, &d cT < \flat{v} \le d\\
\flat{v}^2 cT (2 - cT) \le s^2, &\flat{v} > d\\
\end{cases}\\
\nonumber
\begin{cases}
\flat{v} \le s, &\flat{v} \le d cT\\
\flat{v} \le \frac{s^2 + d^2 c^2 T^2}{2 d cT}, &d cT < \flat{v} \le d\\
\flat{v} \le s \sqrt{\frac{1}{cT (2 - cT)}}, &\flat{v} > d\\
\end{cases}\\
\label{eq:derive_delta_min_pos_root}
\flat{v} &\le s\sqrt{\frac{1}{\iota cT (2 - \iota cT)}}.
\end{align}

\item
Using equation \eqref{eq:r}, it's only valid in range
\begin{align*}
0 \le s - \flat{v}_f \cos\delta_{\min} \le \flat{a},\\
s - \flat{a} \le \flat{v}_f \cos\delta_{\min} \le s,\\
s - sAT \le \sqrt{s^2 - \flat{v}^2 + \flat{v}_f^2} \le s,
\end{align*}
Under condition \eqref{eq:derive_delta_min_pos_root}, then every term is positive, take square.
\begin{align*}
s^2 - s^2 AT (2 - AT) \le s^2 - \flat{v}^2 + \flat{v}_f^2 \le s^2,\\
0 \le \flat{v}^2 - \flat{v}_f^2 \le s^2 AT(2 - AT),\\
0 \le \flat{v}^2 \iota cT(2 - \iota cT) \le s^2 AT(2 - AT),
\end{align*}
Since \eqref{eq:assump_accel_2} and \eqref{eq:assump_friction_2}
\begin{align}
\nonumber
0 \le \flat{v}^2 \le s^2 \frac{AT(2 - AT)}{\iota cT(2 - \iota cT)},\\
\label{eq:delta_min_constraint}
0 \le \flat{v} \le s\sqrt{\frac{AT(2 - AT)}{\iota cT(2 - \iota cT)}},
\end{align}
which is more strict than \eqref{eq:derive_delta_min_pos_root}, because of \eqref{eq:assump_accel_2}. It even simplifies to \eqref{eq:flat_v_max}
\begin{align}
0 \le \flat{v} \le \flat{v}_{\max},
\end{align}

\item
The numerator smaller or equal than the denominator ($\acos$ range)
\begin{align*}
s^2 - \flat{v}^2 + \flat{v}_f^2 &\le \flat{v}_f^2,\\
\flat{v} &\ge s
\end{align*}
\end{itemize}


The $\delta_{\min}$ valid interval/range
\begin{align*}
s \le \flat{v} \le \flat{v}_{\max},\\
\acos\left(\frac{\sqrt{\flat{v}_f^2}}{\flat{v}_f} \right) \le \delta_{\min} \le \acos\left(\frac{\sqrt{s^2 - \flat{v}^2\iota cT(2 - \iota cT)}}{\flat{v}(1 - \iota cT)}\right),\\
\acos(1) \le \delta_{\min} \le \acos\left(\frac{\sqrt{s^2 - s^2 AT(2 - AT)}}{\flat{v}(1 - \iota cT)}\right),\\
0 \le \delta_{\min} \le \acos\left(\frac{s(1 - AT)}{\flat{v}(1 - \iota cT)}\right),\\
0 \le \delta_{\min} \le \acos\left(\frac{s(1 - AT)}{s\sqrt{\frac{AT(2 - AT)}{\iota cT(2 - \iota cT)}} (1 - \iota cT)}\right),\\
0 \le \delta_{\min} \le \acos\left(\sqrt{\frac{\iota cT(2 - \iota cT)}{AT(2 - AT)}} \frac{(1 - AT)}{(1 - \iota cT)}\right),\\
0 \le \delta_{\min} \le \acos\left(\sqrt{\frac{(1 - (1 - \iota cT)^2) (1 - AT)^2}{(1 - (1 - AT)^2) (1 - \iota cT)^2}}\right),\\
0 \le \delta_{\min} \le \delta_{\flat{v}_{\max}},\\
\end{align*}
Note that it simplifies to $\delta_{\opt}$ when approaching the maximum ground speed $\flat{v}_{\max}$.


\subsection{\texorpdfstring{$\delta_{\opt}$}{delta\_opt}}
\eqref{eq:delta_opt}
\begin{align*}
\norm{\projmat{\flat{a}} \vec{\flat{v}}_f} + \norm{\vec{\flat{a}}} &= s \norm{\uvec{\flat{a}}},\\
\flat{v}_f \cos\delta_{\opt} + \flat{a} &= s,\\
\delta_{\opt} &= \acos\left( \frac{s - \flat{a}}{\flat{v}_f} \right).
\end{align*}
Alternatively, using equation \eqref{eq:flat_vf} we find the extended form
\begin{align*}
\delta_{\opt} &= \acos\left( \frac{s - \flat{a}}{\flat{v}_f} \right),\\
&= \acos\left( \frac{s(1 - AT)}{\flat{v}(1 - \iota cT)} \right)
\end{align*}

\begin{itemize}
\item
\eqref{eq:r}, it's only valid in range
\begin{align*}
\flat{v}_f \cos\delta_{\opt} &\le s - a,\\
s - a&\le s - a,
\end{align*}
which is always valid.

\item
The numerator smaller or equal than the denominator ($\acos$ range) + \eqref{eq:assump_accel_1} and \eqref{eq:assump_friction_1}
\begin{align*}
s (1 - AT) &\le \flat{v}(1 - \iota cT),\\
\flat{v} &\ge s \frac{(1 - AT)}{(1 - \iota cT)},
\end{align*}
\end{itemize}

The $\delta_{\opt}$ valid interval/range
\begin{align*}
s \frac{(1 - AT)}{(1 - \iota cT)} &\le \flat{v},\\
\acos\left( \frac{s(1 - AT)}{s(1 - AT)} \right) &\le \delta_{\opt},\\
0 &\le \delta_{\opt},
\end{align*}


\subsection{\texorpdfstring{$\delta_{\max}$}{delta\_max}}
\eqref{eq:delta_max}
\begin{align*}
\norm{\vec{\flat{v}}_f + \vec{\flat{a}}}^2 &= \norm{\vec{\flat{v}}}^2,\\
(\vec{\flat{v}}_f + \vec{\flat{a}})^T (\vec{\flat{v}}_f + \vec{\flat{a}}) &= \vec{\flat{v}}^T \vec{\flat{v}},\\
2 \vec{\flat{a}}^T \vec{\flat{v}}_f + \vec{\flat{a}}^T \vec{\flat{a}} &= \vec{\flat{v}}^T \vec{\flat{v}} - \vec{\flat{v}}_f^T \vec{\flat{v}}_f,\\
2 \flat{a} \flat{v}_f \cos\delta_{\max} + \flat{a}^2 &= \flat{v}^2 - \flat{v}_f^2,\\
\delta_{\max} &= \acos\left( \frac{\flat{v}^2 - \flat{v}_f^2 - \flat{a}^2}{2 \flat{a} \flat{v}_f} \right).
\end{align*}
Alternatively, using equation \eqref{eq:flat_vf} we find the extended form
\begin{align*}
\delta_{\max} &= \acos\left( \frac{\flat{v}^2 - \flat{v}_f^2 - \flat{a}^2}{2 \flat{a} \flat{v}_f} \right),\\
&= \acos\left( \frac{\flat{v}^2 \iota cT(2 - \iota cT) - s^2 A^2 T^2}{2 s \flat{v} AT(1 - \iota cT)} \right),\\
&= \acos\left( \frac{\flat{v} \flat{v} (\iota cT)(2 - \iota cT) - ss (AT)(AT)}{s \flat{v} (AT)(2 - \iota cT) - s \flat{v} (AT)(\iota cT)} \right),
\end{align*}

\begin{itemize}
\item
\eqref{eq:r}, it's only valid in range + \eqref{eq:assump_accel_2} and \eqref{eq:assump_friction_2}
\begin{align*}
\flat{v}_f \cos\delta_{\max} &\le s - a,\\
\frac{\flat{v}^2 - \flat{v}_f^2 - \flat{a}^2}{2 \flat{a}}&\le s - a,\\
\flat{v}^2 - \flat{v}_f^2 &\le 2 a s - a^2,\\
\flat{v}^2 \iota cT(2 - \iota cT)&\le s^2 AT(2 - AT),\\
\flat{v} &\le s \sqrt{\frac{AT(2 - AT)}{\iota cT(2 - \iota cT)}},
\end{align*}

\item
The numerator can be both positive and negative. Numerator negative, in this case the numerator is smaller or equal than the denominator ($\acos$ range) when
\begin{align*}
-\flat{v}^2 + \flat{v}_f^2 + \flat{a}^2 &\le 2 \flat{a} \flat{v}_f,\\
-\flat{v}^2 \iota cT (2 - \iota cT) + \flat{a}^2 &\le 2 \flat{a} \flat{v} (1 - \iota cT),\\
\flat{a}^2 + 2 \flat{a} (\flat{v} \iota cT) + (\flat{v} \iota cT)^2 &\le 2 \flat{a} \flat{v} + 2 \flat{v}^2 \iota cT,\\
(\flat{a} + \flat{v} \iota cT)^2 &\le 2 \flat{v} (\flat{a} + \flat{v} \iota cT),\\
\end{align*}
since \eqref{eq:assump_friction_1}
\begin{align*}
\flat{a} + \flat{v} \iota cT &\le 2 \flat{v},\\
\flat{v} &\ge s \frac{AT}{2 - \iota cT},\\
\end{align*}

\item
Numerator positive, in this case the numerator is smaller or equal than the denominator ($\acos$ range) when
\begin{align}
\nonumber
\flat{v}^2 - \flat{v}_f^2 - \flat{a}^2 &\le 2 \flat{a} \flat{v}_f,\\
\nonumber
\flat{v}^2 \iota cT (2 - \iota cT) - \flat{a}^2 &\le 2 \flat{a} \flat{v} (1 - \iota cT),\\
\nonumber
- \flat{a}^2 + 2 \flat{a} (\flat{v} \iota cT) - (\flat{v} \iota cT)^2 &\le 2 \flat{a} \flat{v} - 2 \flat{v}^2 \iota cT,\\
\label{eq:delta_max_constraint}
-(\flat{a} - \flat{v} \iota cT)^2 &\le 2 \flat{v} (\flat{a} - \flat{v} \iota cT),
\end{align}
$\flat{a} - \flat{v} \iota cT$ can be both positive and negative. When $\flat{a} - \flat{v} \iota cT \le 0$ \eqref{eq:delta_max_constraint} becomes with \eqref{eq:assump_friction_1}
\begin{align*}
-(\flat{a} - \flat{v} \iota cT) &\ge 2 \flat{v},\\
\flat{v} &\le -s \frac{AT}{2 - \iota cT},
\end{align*}
which is never true, because $\flat{v} \ge 0$. When $\flat{a} - \flat{v} \iota cT \ge 0$ \eqref{eq:delta_max_constraint} becomes with \eqref{eq:assump_friction_1}
\begin{align*}
-(\flat{a} - \flat{v} \iota cT) &\le 2 \flat{v},
\end{align*}
which is always true. Hence,
\begin{align*}
\flat{a} - \flat{v} \iota cT &\ge 0,\\
\flat{v} &\le s \frac{AT}{\iota cT},
\end{align*}
\end{itemize}

The $\delta_{\max}$ valid interval/range
\begin{align*}
s \frac{AT}{2 - \iota cT} \le \flat{v} \le
\begin{cases}
s \frac{AT}{\iota cT}, & AT \le \iota cT,\\
s \sqrt{\frac{AT(2 - AT)}{\iota cT(2 - \iota cT)}}, &AT > \iota cT,
\end{cases}\\
\acos\left( \frac{\frac{s^2A^2T^2\iota cT}{2 - \iota cT} - s^2A^2T^2}{s^2A^2T^2 - \frac{s^2A^2T^2\iota cT}{2 - \iota cT}} \right) \ge \delta_{\max} \ge
\begin{cases}
\acos\left( \frac{\frac{s^2A^2T^2(2 - \iota cT)}{\iota cT} - s^2A^2T^2}{\frac{s^2A^2T^2(2 - \iota cT)}{\iota cT} - s^2A^2T^2} \right), & AT \le \iota cT,\\
\acos\left( \frac{s^2 AT(2 - AT) - s^2 A^2 T^2}{2 s \flat{v} AT(1 - \iota cT)} \right), &AT > \iota cT,
\end{cases}\\
\acos(-1) \ge \delta_{\max} \ge
\begin{cases}
\acos(1), & AT \le \iota cT,\\
\acos\left( \frac{2 s s AT(1 - AT)}{2 s \flat{v} AT(1 - \iota cT)} \right), &AT > \iota cT,
\end{cases}\\
\pi \ge \delta_{\max} \ge
\begin{cases}
0, & AT \le \iota cT,\\
\acos\left( \frac{s(1 - AT)}{\flat{v} (1 - \iota cT)} \right), &AT > \iota cT,
\end{cases}
\end{align*}
Note that it simplifies to $\delta_{\opt}$ when approaching the maximum ground speed $\flat{v}_{\max}$.


\subsection{\texorpdfstring{$\bar{\delta}_{\min}$}{bar delta\_min}}
\eqref{eq:delta_bar_min}
\begin{align*}
\norm{\projmat{\flat{v}} \left( \vec{\flat{v}}_f + (s - \flat{v}_f \cos\bar{\delta}_{\min})\uvec{\flat{a}} \right)} &= \norm{\vec{\flat{v}}},\\
\norm{\flat{v}_f \uvec{\flat{v}} + (s - \flat{v}_f \cos\bar{\delta}_{\min})\cos\bar{\delta}_{\min}\uvec{\flat{v}}} &= \flat{v},\\
(\flat{v}_f + s\cos\bar{\delta}_{\min} - \flat{v}_f \cos^2\bar{\delta}_{\min}) \norm{\uvec{\flat{v}}} &= \flat{v},\\
s\cos\bar{\delta}_{\min} - \flat{v}_f \cos^2\bar{\delta}_{\min} &= \flat{v} - \flat{v}_f, \qquad D = s^2 - 4 \flat{v}_f (\flat{v} - \flat{v}_f),\\
\bar{\delta}_{\min} &= \acos\left( \frac{s \pm \sqrt{s^2 - 4 \flat{v}_f (\flat{v} - \flat{v}_f)}}{2 \flat{v}_f} \right).\\
\end{align*}
Since $s \ge \flat{a}$ and the acceleration should not be greater than the full acceleration $\flat{a}$, mathematically
\begin{align*}
0 \le s - \flat{v}_f \cos\bar{\delta}_{\min} = s / 2 \mp \sqrt{s^2 - 4 \flat{v}_f (\flat{v} - \flat{v}_f)} \Big/ 2 \le \flat{a}.
\end{align*}
We find that this only holds for
\begin{align*}
\bar{\delta}_{\min} &= \acos\left( \frac{s + \sqrt{s^2 - 4 \flat{v}_f (\flat{v} - \flat{v}_f)}}{2 \flat{v}_f} \right).
\end{align*}


\subsection{\texorpdfstring{$\bar{\delta}_{\max}$}{bar delta\_max}}
\eqref{eq:delta_bar_max}
\begin{align*}
\norm{\projmat{\flat{v}}\left( \vec{\flat{v}}_f + \vec{\flat{a}} \right)} &= \norm{\vec{\flat{v}}},\\
\norm{\flat{v}_f \uvec{\flat{v}} + \flat{a} \cos\bar{\delta}_{\max}\uvec{\flat{v}}} &= \flat{v},\\
(\flat{v}_f + \flat{a} \cos\bar{\delta}_{\max}) \norm{\uvec{\flat{v}}} &= \flat{v},\\
\flat{a} \cos\bar{\delta}_{\max} &= \flat{v} - \flat{v}_f,\\
\bar{\delta}_{\max} &= \acos\left( \frac{\flat{v} - \flat{v}_f}{\flat{a}} \right).
\end{align*}
\begin{align*}
\bar{\delta}_{\max} &= \acos\left( \frac{\flat{v} - \flat{v}_f}{\flat{a}} \right),\\
&= \acos\left( \frac{\flat{v} \iota cT}{sAT} \right).
\end{align*}

\begin{itemize}
\item
\eqref{eq:r}, it's only valid in range
\begin{align*}
\flat{v}_f \cos\bar{\delta}_{\max} &\le s - a,\\
\frac{\flat{v}^2 \iota cT(1 - \iota cT)}{sAT} &\le s(1 - AT),\\
\flat{v}^2 \iota cT(1 - \iota cT) &\le s^2 AT(1 - AT),\\
\flat{v} &\le s \sqrt{\frac{AT(1 - AT)}{\iota cT(1 - \iota cT)}},
\end{align*}

\item
The numerator is smaller or equal than the denominator ($\acos$ range) when
\begin{align*}
\flat{v} \iota cT \le sAT,\\
\flat{v} \le s \frac{AT}{\iota cT},
\end{align*}
\end{itemize}

The $\bar{\delta}_{\max}$ valid interval/range
\begin{align*}
0 \le \flat{v} \le
\begin{cases}
s \frac{AT}{\iota cT}, & AT \le \iota cT,\\
s \sqrt{\frac{AT(1 - AT)}{\iota cT(1 - \iota cT)}}, &AT > \iota cT,
\end{cases}\\
\acos(0) \ge \bar{\delta}_{\max} \ge
\begin{cases}
\acos(1), & AT \le \iota cT,\\
\acos\left( \frac{\iota cT}{AT} \sqrt{\frac{AT(1 - AT)}{\iota cT(1 - \iota cT)}} \right), &AT > \iota cT,
\end{cases}\\
\frac{\pi}{2} \ge \bar{\delta}_{\max} \ge
\begin{cases}
0, & AT \le \iota cT,\\
\acos\left( \sqrt{\frac{\iota cT(1 - AT)}{AT(1 - \iota cT)}} \right), &AT > \iota cT,
\end{cases}
\end{align*}


\subsection{\texorpdfstring{$\flat{v}_{\max}$}{flat v\_max}}
\label{app:derive_flat_v_max}
We find the \emph{maximum ground speed} \eqref{eq:flat_v_max} by
\begin{align}
\nonumber
\eqref{eq:delta_max} &= \eqref{eq:delta_min},\\
\nonumber
\frac{(\flat{v}^2 - \flat{v}_f^2) - \flat{a}^2}{2 \flat{a}\flat{v}_f} &= \frac{\sqrt{s^2 - (\flat{v}^2 - \flat{v}_f^2)}}{\flat{v}_f},\\
\nonumber
\left((\flat{v}^2 - \flat{v}_f^2) - \flat{a}^2\right)^2 &= 4 \flat{a}^2(s^2 - (\flat{v}^2 - \flat{v}_f^2)),\\
\nonumber
(\flat{v}^2 - \flat{v}_f^2)^2 - 2 \flat{a}^2(\flat{v}^2 - \flat{v}_f^2) + \flat{a}^4 &= 4 \flat{a}^2s^2 - 4 \flat{a}^2 (\flat{v}^2 - \flat{v}_f^2),\\
\nonumber
(\flat{v}^2 - \flat{v}_f^2)^2 + 2 \flat{a}^2(\flat{v}^2 - \flat{v}_f^2) + \flat{a}^4 &= 4 \flat{a}^2 s^2,\\
\nonumber
\left((\flat{v}^2 - \flat{v}_f^2) + \flat{a}^2\right)^2 &= 4 \flat{a}^2s^2,\\
\label{eq:maxgs}
\flat{v}^2 - \flat{v}_f^2 &= 2 \flat{a} s - \flat{a}^2,\\
\label{eq:maxgs_iota}
\flat{v}^2 \iota cT (2 - \iota cT) &= s^2 AT(2 - AT),\\
\begin{cases}
\flat{v}^2 = s^2 AT(2 - AT), &\flat{v} \le d cT,\\
d cT (2 \flat{v} - d cT) = s^2 AT(2 - AT), &d cT < \flat{v} \le d,\\
\flat{v}^2 cT (2 - cT) = s^2 AT(2 - AT), &\flat{v} > d,\\
\end{cases}\\
\flat{v} =
\begin{cases}
s \sqrt{AT(2 - AT)}, &\flat{v} \le d cT,\\
\frac{s^2 AT(2 - AT) + d^2 c^2 T^2}{2 d cT}, &d cT < \flat{v} \le d,\\
s \sqrt{\frac{AT(2 - AT)}{cT (2 - cT)}}, &\flat{v} > d\\
\end{cases}
\end{align}\\
assuming that $\flat{v} \ge d$ and substituting $\flat{v}_f$ and $\flat{a}$ using equation \eqref{eq:flat_vf} and \eqref{eq:sAT} respectively, we get
\begin{align*}
\flat{v}^2 cT(2 - cT) &= s^2 AT(2 - AT),\\
\flat{v}_{\max} &= s \sqrt{\frac{A(2 - AT)}{c(2 - cT)}}.
\end{align*}
As one might notice, the angle $\delta_{\opt}$ also meets at this point, because
\begin{align*}
\eqref{eq:delta_max} &= \eqref{eq:delta_opt},\\
\frac{(\flat{v}^2 - \flat{v}_f^2) - \flat{a}^2}{2 \flat{a} \flat{v}_f} &= \frac{s - \flat{a}}{\flat{v}_f},\\
\flat{v}^2 - \flat{v}_f^2 &= 2 \flat{a} s - \flat{a}^2
\end{align*}
is identical to equation \eqref{eq:maxgs}.


\subsubsection{\texorpdfstring{$\flat{v}_{\max}$ alternative}{flat v\_max alternative}}
\label{app:derive_flat_v_max_alternative}
To solve for $\flat{v}_{\max}$, equate $\eqref{eq:accel_opt_ground} = \flat{v}$.
\begin{align*}
\flat{v} &= \sqrt{\left((1 - \iota cT)\flat{v} \right)^2 + 2s\flat{a} - \flat{a}^2}\\
\flat{v}^2 &= \flat{v}^2\left(1^2 + (\iota cT)^2 - 2\iota cT \right)^2 + s^2(2AT - A^2T^2)\\
\flat{v}^2(\iota cT(\iota cT - 2)) &= -s^2(AT(2 - AT))\\
\flat{v}^2\iota cT(2 - \iota cT) &= s^2AT(2 - AT),
\end{align*}
which is identical to \eqref{eq:maxgs_iota}.\\
Assuming $\flat{v} \ge d$, this becomes the same as above, \eqref{eq:flat_v_max}.


\subsection{\texorpdfstring{$x_z\{k\}$}{x\_z k}}
\label{app:derive_x_zk}
To derive $x_z\{k\}$ in terms of $x_z\{0\}$ and $v_z\{0\}$,
\begin{align*}
x_z\{1\} &= x_z\{0\} + \left(v_z\{0\} - \frac{gT_p}{2} \right) T_p\\
x_z\{2\} &= x_z\{1\} + \left(v_z\{1\} - \frac{gT_p}{2} \right) T_p\\
&= x_z\{0\} + \left(v_z\{0\} - \frac{gT_p}{2} \right) T_p + \left(v_z\{0\} - \round{gT_p} - \frac{gT_p}{2} \right) T_p\\
&= x_z\{0\} + 2\left(v_z\{0\} - \frac{gT_p}{2} \right) T_p - \round{gT_p} T_p\\
x_z\{3\} &= x_z\{1\} + 2\left(v_z\{1\} - \frac{gT_p}{2} \right) T_p - \round{gT_p} T_p\\
&= x_z\{0\} + \left(v_z\{0\} - \frac{gT_p}{2} \right) T_p + 2\left(v_z\{0\} - \round{gT_p} - \frac{gT_p}{2} \right) T_p - \round{gT_p} T_p\\
&= x_z\{0\} + 3\left(v_z\{0\} - \frac{gT_p}{2} \right) T_p - 3\round{gT_p} T_p\\
x_z\{4\} &= x_z\{1\} + 3\left(v_z\{1\} - \frac{gT_p}{2} \right) T_p - 3\round{gT_p} T_p\\
&= x_z\{0\} + \left(v_z\{0\} - \frac{gT_p}{2} \right) T_p + 3\left(v_z\{0\} - \round{gT_p} - \frac{gT_p}{2} \right) T_p - 3\round{gT_p} T_p\\
&= x_z\{0\} + 4\left(v_z\{0\} - \frac{gT_p}{2} \right) T_p - 6\round{gT_p} T_p\\
x_z\{5\} &= x_z\{1\} + 4\left(v_z\{1\} - \frac{gT_p}{2} \right) T_p - 6\round{gT_p} T_p\\
&= x_z\{0\} + \left(v_z\{0\} - \frac{gT_p}{2} \right) T_p + 4\left(v_z\{0\} - \round{gT_p} - \frac{gT_p}{2} \right) T_p - 6\round{gT_p} T_p\\
&= x_z\{0\} + 5\left(v_z\{0\} - \frac{gT_p}{2} \right) T_p - 10\round{gT_p} T_p\\
x_z\{k\} &= x_z\{0\} + k\left(v_z\{0\} - \frac{gT_p}{2} \right) T_p - \frac{k(k - 1)}{2} \round{gT_p} T_p\\
&= x_z\{0\} + k\left(v_z\{0\} - \frac{gT_p}{2} \right) T_p + \frac{k}{2} \round{gT_p} T_p - \frac{k^2}{2} \round{gT_p} T_p\\
&= x_z\{0\} + k\left(v_z\{0\} - \frac{gT_p}{2} + \frac{\round{gT_p}}{2} \right) T_p - \frac{k^2}{2} \round{gT_p} T_p,
\end{align*}
deriving \eqref{eq:x_zk}.

	\section{Field of view}
\label{app:fov}

\begin{figure}[H]
	\centering
	\begin{subfigure}[t]{.5\textwidth}
		\centering
		\setlength\figureheight{5.5cm} 
		\setlength\figurewidth{5.5cm}
		\includetikz{tikz/fov_1D}
		\caption{}
	\end{subfigure}%
	\begin{subfigure}[t]{.5\textwidth}
		\centering
		\setlength\figureheight{5.5cm} 
		\setlength\figurewidth{5.5cm}
		\includetikz{tikz/fov_1D_}
		\caption{}
	\end{subfigure}%
	\caption{TODO}
	\label{fig:fov}
\end{figure}
Field of view $\Lambda_x$\footnote{The field of view $\Lambda_x$ is expressed in radians, while \texttt{cg\_fov} in degrees.} (\texttt{cg\_fov})
\begin{align*}
w &= 2\tan\frac{\Lambda_x}{2}\\
S &= \frac{W}{w} = \frac{W}{2\tan\frac{\Lambda_x}{2}}
\end{align*}
with $W = 640$ the width of the default screen \texttt{SCREEN\_WIDTH}.

$\alpha$ is the desired yaw angle and $\gamma$ current viewing direction
\begin{align*}
x &=
\begin{cases}
-\tan\frac{\Lambda_x}{2}, & \gamma - \alpha  \le -\frac{\Lambda_x}{2}, \\
\tan(\gamma - \alpha), & \abs{\gamma - \alpha} < \frac{\Lambda_x}{2},\\
\tan\frac{\Lambda_x}{2}, & \gamma - \alpha \ge \frac{\Lambda_x}{2},
\end{cases}
\end{align*}

\begin{align*}
X &= \frac{W}{2} + \frac{W}{2}\frac{x}{\tan\frac{\Lambda_x}{2}},\\
X &=
\begin{cases}
0, & \gamma - \alpha  \le -\frac{\Lambda_x}{2}, \\
\frac{W}{2} + \frac{W}{2}\frac{\tan(\gamma - \alpha)}{\tan\frac{\Lambda}{2}}, & \abs{\gamma - \alpha} < \frac{\Lambda_x}{2},\\
W, & \gamma - \alpha \ge \frac{\Lambda_x}{2},
\end{cases}                                 
\end{align*}

\begin{align*}
X_2 - X_1 &= - \frac{\tan(\alpha_{<}-\gamma)}{\tan\left(\frac{\texttt{cg\_fov}}{2}\right)} \frac{\texttt{SCREEN\_WIDTH}}{2} + \frac{\tan(\alpha_{>}-\gamma)}{\tan\left(\frac{\texttt{cg\_fov}}{2}\right)} \frac{\texttt{SCREEN\_WIDTH}}{2}\\
width &= \frac{\texttt{SCREEN\_WIDTH}}{2} \frac{\tan(\alpha_{>}-\gamma) - \tan(\alpha_{<}-\gamma)}{\tan\left(\frac{\texttt{cg\_fov}}{2}\right)}
\end{align*}

\begin{align*}
y &= \tan(\beta-\rho)\\
Y &= \frac{H}{2} + Sy\\
y &= 240 + 320\frac{\tan(\beta-\rho)}{\tan\frac{\Lambda}{2}}
\end{align*}

\end{appendices}

\bibliographystyle{alpha}
\bibliography{references}

\end{document}
